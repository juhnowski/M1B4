\documentclass[11pt]{article}

    \usepackage[breakable]{tcolorbox}
    \usepackage{parskip} % Stop auto-indenting (to mimic markdown behaviour)
    

    % Basic figure setup, for now with no caption control since it's done
    % automatically by Pandoc (which extracts ![](path) syntax from Markdown).
    \usepackage{graphicx}
    % Maintain compatibility with old templates. Remove in nbconvert 6.0
    \let\Oldincludegraphics\includegraphics
    % Ensure that by default, figures have no caption (until we provide a
    % proper Figure object with a Caption API and a way to capture that
    % in the conversion process - todo).
    \usepackage{caption}
    \DeclareCaptionFormat{nocaption}{}
    \captionsetup{format=nocaption,aboveskip=0pt,belowskip=0pt}

    \usepackage{float}
    \floatplacement{figure}{H} % forces figures to be placed at the correct location
    \usepackage{xcolor} % Allow colors to be defined
    \usepackage{enumerate} % Needed for markdown enumerations to work
    \usepackage{geometry} % Used to adjust the document margins
    \usepackage{amsmath} % Equations
    \usepackage{amssymb} % Equations
    \usepackage{textcomp} % defines textquotesingle
    % Hack from http://tex.stackexchange.com/a/47451/13684:
    \AtBeginDocument{%
        \def\PYZsq{\textquotesingle}% Upright quotes in Pygmentized code
    }
    \usepackage{upquote} % Upright quotes for verbatim code
    \usepackage{eurosym} % defines \euro

    \usepackage{iftex}
    \ifPDFTeX
        \usepackage[T1]{fontenc}
        \IfFileExists{alphabeta.sty}{
              \usepackage{alphabeta}
          }{
              \usepackage[mathletters]{ucs}
              \usepackage[utf8x]{inputenc}
          }
    \else
        \usepackage{fontspec}
        \usepackage{unicode-math}
    \fi

    \usepackage{fancyvrb} % verbatim replacement that allows latex
    \usepackage{grffile} % extends the file name processing of package graphics 
                         % to support a larger range
    \makeatletter % fix for old versions of grffile with XeLaTeX
    \@ifpackagelater{grffile}{2019/11/01}
    {
      % Do nothing on new versions
    }
    {
      \def\Gread@@xetex#1{%
        \IfFileExists{"\Gin@base".bb}%
        {\Gread@eps{\Gin@base.bb}}%
        {\Gread@@xetex@aux#1}%
      }
    }
    \makeatother
    \usepackage[Export]{adjustbox} % Used to constrain images to a maximum size
    \adjustboxset{max size={0.9\linewidth}{0.9\paperheight}}

    % The hyperref package gives us a pdf with properly built
    % internal navigation ('pdf bookmarks' for the table of contents,
    % internal cross-reference links, web links for URLs, etc.)
    \usepackage{hyperref}
    % The default LaTeX title has an obnoxious amount of whitespace. By default,
    % titling removes some of it. It also provides customization options.
    \usepackage{titling}
    \usepackage{longtable} % longtable support required by pandoc >1.10
    \usepackage{booktabs}  % table support for pandoc > 1.12.2
    \usepackage{array}     % table support for pandoc >= 2.11.3
    \usepackage{calc}      % table minipage width calculation for pandoc >= 2.11.1
    \usepackage[inline]{enumitem} % IRkernel/repr support (it uses the enumerate* environment)
    \usepackage[normalem]{ulem} % ulem is needed to support strikethroughs (\sout)
                                % normalem makes italics be italics, not underlines
    \usepackage{mathrsfs}
    
  	\usepackage{ucs} 
	\usepackage[utf8x]{inputenc} % Включаем поддержку UTF8  
	\usepackage[russian]{babel}  % Включаем пакет для поддержки русского языка 
    

    
    % Colors for the hyperref package
    \definecolor{urlcolor}{rgb}{0,.145,.698}
    \definecolor{linkcolor}{rgb}{.71,0.21,0.01}
    \definecolor{citecolor}{rgb}{.12,.54,.11}

    % ANSI colors
    \definecolor{ansi-black}{HTML}{3E424D}
    \definecolor{ansi-black-intense}{HTML}{282C36}
    \definecolor{ansi-red}{HTML}{E75C58}
    \definecolor{ansi-red-intense}{HTML}{B22B31}
    \definecolor{ansi-green}{HTML}{00A250}
    \definecolor{ansi-green-intense}{HTML}{007427}
    \definecolor{ansi-yellow}{HTML}{DDB62B}
    \definecolor{ansi-yellow-intense}{HTML}{B27D12}
    \definecolor{ansi-blue}{HTML}{208FFB}
    \definecolor{ansi-blue-intense}{HTML}{0065CA}
    \definecolor{ansi-magenta}{HTML}{D160C4}
    \definecolor{ansi-magenta-intense}{HTML}{A03196}
    \definecolor{ansi-cyan}{HTML}{60C6C8}
    \definecolor{ansi-cyan-intense}{HTML}{258F8F}
    \definecolor{ansi-white}{HTML}{C5C1B4}
    \definecolor{ansi-white-intense}{HTML}{A1A6B2}
    \definecolor{ansi-default-inverse-fg}{HTML}{FFFFFF}
    \definecolor{ansi-default-inverse-bg}{HTML}{000000}

    % common color for the border for error outputs.
    \definecolor{outerrorbackground}{HTML}{FFDFDF}

    % commands and environments needed by pandoc snippets
    % extracted from the output of `pandoc -s`
    \providecommand{\tightlist}{%
      \setlength{\itemsep}{0pt}\setlength{\parskip}{0pt}}
    \DefineVerbatimEnvironment{Highlighting}{Verbatim}{commandchars=\\\{\}}
    % Add ',fontsize=\small' for more characters per line
    \newenvironment{Shaded}{}{}
    \newcommand{\KeywordTok}[1]{\textcolor[rgb]{0.00,0.44,0.13}{\textbf{{#1}}}}
    \newcommand{\DataTypeTok}[1]{\textcolor[rgb]{0.56,0.13,0.00}{{#1}}}
    \newcommand{\DecValTok}[1]{\textcolor[rgb]{0.25,0.63,0.44}{{#1}}}
    \newcommand{\BaseNTok}[1]{\textcolor[rgb]{0.25,0.63,0.44}{{#1}}}
    \newcommand{\FloatTok}[1]{\textcolor[rgb]{0.25,0.63,0.44}{{#1}}}
    \newcommand{\CharTok}[1]{\textcolor[rgb]{0.25,0.44,0.63}{{#1}}}
    \newcommand{\StringTok}[1]{\textcolor[rgb]{0.25,0.44,0.63}{{#1}}}
    \newcommand{\CommentTok}[1]{\textcolor[rgb]{0.38,0.63,0.69}{\textit{{#1}}}}
    \newcommand{\OtherTok}[1]{\textcolor[rgb]{0.00,0.44,0.13}{{#1}}}
    \newcommand{\AlertTok}[1]{\textcolor[rgb]{1.00,0.00,0.00}{\textbf{{#1}}}}
    \newcommand{\FunctionTok}[1]{\textcolor[rgb]{0.02,0.16,0.49}{{#1}}}
    \newcommand{\RegionMarkerTok}[1]{{#1}}
    \newcommand{\ErrorTok}[1]{\textcolor[rgb]{1.00,0.00,0.00}{\textbf{{#1}}}}
    \newcommand{\NormalTok}[1]{{#1}}
    
    % Additional commands for more recent versions of Pandoc
    \newcommand{\ConstantTok}[1]{\textcolor[rgb]{0.53,0.00,0.00}{{#1}}}
    \newcommand{\SpecialCharTok}[1]{\textcolor[rgb]{0.25,0.44,0.63}{{#1}}}
    \newcommand{\VerbatimStringTok}[1]{\textcolor[rgb]{0.25,0.44,0.63}{{#1}}}
    \newcommand{\SpecialStringTok}[1]{\textcolor[rgb]{0.73,0.40,0.53}{{#1}}}
    \newcommand{\ImportTok}[1]{{#1}}
    \newcommand{\DocumentationTok}[1]{\textcolor[rgb]{0.73,0.13,0.13}{\textit{{#1}}}}
    \newcommand{\AnnotationTok}[1]{\textcolor[rgb]{0.38,0.63,0.69}{\textbf{\textit{{#1}}}}}
    \newcommand{\CommentVarTok}[1]{\textcolor[rgb]{0.38,0.63,0.69}{\textbf{\textit{{#1}}}}}
    \newcommand{\VariableTok}[1]{\textcolor[rgb]{0.10,0.09,0.49}{{#1}}}
    \newcommand{\ControlFlowTok}[1]{\textcolor[rgb]{0.00,0.44,0.13}{\textbf{{#1}}}}
    \newcommand{\OperatorTok}[1]{\textcolor[rgb]{0.40,0.40,0.40}{{#1}}}
    \newcommand{\BuiltInTok}[1]{{#1}}
    \newcommand{\ExtensionTok}[1]{{#1}}
    \newcommand{\PreprocessorTok}[1]{\textcolor[rgb]{0.74,0.48,0.00}{{#1}}}
    \newcommand{\AttributeTok}[1]{\textcolor[rgb]{0.49,0.56,0.16}{{#1}}}
    \newcommand{\InformationTok}[1]{\textcolor[rgb]{0.38,0.63,0.69}{\textbf{\textit{{#1}}}}}
    \newcommand{\WarningTok}[1]{\textcolor[rgb]{0.38,0.63,0.69}{\textbf{\textit{{#1}}}}}
    
    
    % Define a nice break command that doesn't care if a line doesn't already
    % exist.
    \def\br{\hspace*{\fill} \\* }
    % Math Jax compatibility definitions
    \def\gt{>}
    \def\lt{<}
    \let\Oldtex\TeX
    \let\Oldlatex\LaTeX
    \renewcommand{\TeX}{\textrm{\Oldtex}}
    \renewcommand{\LaTeX}{\textrm{\Oldlatex}}
    % Document parameters
    % Document title
    \title{Практическое занятие 2.2 Создание новых материалов. Тканевая инженерия. Компенсационные материалы}
    
    
    
    
    
% Pygments definitions
\makeatletter
\def\PY@reset{\let\PY@it=\relax \let\PY@bf=\relax%
    \let\PY@ul=\relax \let\PY@tc=\relax%
    \let\PY@bc=\relax \let\PY@ff=\relax}
\def\PY@tok#1{\csname PY@tok@#1\endcsname}
\def\PY@toks#1+{\ifx\relax#1\empty\else%
    \PY@tok{#1}\expandafter\PY@toks\fi}
\def\PY@do#1{\PY@bc{\PY@tc{\PY@ul{%
    \PY@it{\PY@bf{\PY@ff{#1}}}}}}}
\def\PY#1#2{\PY@reset\PY@toks#1+\relax+\PY@do{#2}}

\@namedef{PY@tok@w}{\def\PY@tc##1{\textcolor[rgb]{0.73,0.73,0.73}{##1}}}
\@namedef{PY@tok@c}{\let\PY@it=\textit\def\PY@tc##1{\textcolor[rgb]{0.25,0.50,0.50}{##1}}}
\@namedef{PY@tok@cp}{\def\PY@tc##1{\textcolor[rgb]{0.74,0.48,0.00}{##1}}}
\@namedef{PY@tok@k}{\let\PY@bf=\textbf\def\PY@tc##1{\textcolor[rgb]{0.00,0.50,0.00}{##1}}}
\@namedef{PY@tok@kp}{\def\PY@tc##1{\textcolor[rgb]{0.00,0.50,0.00}{##1}}}
\@namedef{PY@tok@kt}{\def\PY@tc##1{\textcolor[rgb]{0.69,0.00,0.25}{##1}}}
\@namedef{PY@tok@o}{\def\PY@tc##1{\textcolor[rgb]{0.40,0.40,0.40}{##1}}}
\@namedef{PY@tok@ow}{\let\PY@bf=\textbf\def\PY@tc##1{\textcolor[rgb]{0.67,0.13,1.00}{##1}}}
\@namedef{PY@tok@nb}{\def\PY@tc##1{\textcolor[rgb]{0.00,0.50,0.00}{##1}}}
\@namedef{PY@tok@nf}{\def\PY@tc##1{\textcolor[rgb]{0.00,0.00,1.00}{##1}}}
\@namedef{PY@tok@nc}{\let\PY@bf=\textbf\def\PY@tc##1{\textcolor[rgb]{0.00,0.00,1.00}{##1}}}
\@namedef{PY@tok@nn}{\let\PY@bf=\textbf\def\PY@tc##1{\textcolor[rgb]{0.00,0.00,1.00}{##1}}}
\@namedef{PY@tok@ne}{\let\PY@bf=\textbf\def\PY@tc##1{\textcolor[rgb]{0.82,0.25,0.23}{##1}}}
\@namedef{PY@tok@nv}{\def\PY@tc##1{\textcolor[rgb]{0.10,0.09,0.49}{##1}}}
\@namedef{PY@tok@no}{\def\PY@tc##1{\textcolor[rgb]{0.53,0.00,0.00}{##1}}}
\@namedef{PY@tok@nl}{\def\PY@tc##1{\textcolor[rgb]{0.63,0.63,0.00}{##1}}}
\@namedef{PY@tok@ni}{\let\PY@bf=\textbf\def\PY@tc##1{\textcolor[rgb]{0.60,0.60,0.60}{##1}}}
\@namedef{PY@tok@na}{\def\PY@tc##1{\textcolor[rgb]{0.49,0.56,0.16}{##1}}}
\@namedef{PY@tok@nt}{\let\PY@bf=\textbf\def\PY@tc##1{\textcolor[rgb]{0.00,0.50,0.00}{##1}}}
\@namedef{PY@tok@nd}{\def\PY@tc##1{\textcolor[rgb]{0.67,0.13,1.00}{##1}}}
\@namedef{PY@tok@s}{\def\PY@tc##1{\textcolor[rgb]{0.73,0.13,0.13}{##1}}}
\@namedef{PY@tok@sd}{\let\PY@it=\textit\def\PY@tc##1{\textcolor[rgb]{0.73,0.13,0.13}{##1}}}
\@namedef{PY@tok@si}{\let\PY@bf=\textbf\def\PY@tc##1{\textcolor[rgb]{0.73,0.40,0.53}{##1}}}
\@namedef{PY@tok@se}{\let\PY@bf=\textbf\def\PY@tc##1{\textcolor[rgb]{0.73,0.40,0.13}{##1}}}
\@namedef{PY@tok@sr}{\def\PY@tc##1{\textcolor[rgb]{0.73,0.40,0.53}{##1}}}
\@namedef{PY@tok@ss}{\def\PY@tc##1{\textcolor[rgb]{0.10,0.09,0.49}{##1}}}
\@namedef{PY@tok@sx}{\def\PY@tc##1{\textcolor[rgb]{0.00,0.50,0.00}{##1}}}
\@namedef{PY@tok@m}{\def\PY@tc##1{\textcolor[rgb]{0.40,0.40,0.40}{##1}}}
\@namedef{PY@tok@gh}{\let\PY@bf=\textbf\def\PY@tc##1{\textcolor[rgb]{0.00,0.00,0.50}{##1}}}
\@namedef{PY@tok@gu}{\let\PY@bf=\textbf\def\PY@tc##1{\textcolor[rgb]{0.50,0.00,0.50}{##1}}}
\@namedef{PY@tok@gd}{\def\PY@tc##1{\textcolor[rgb]{0.63,0.00,0.00}{##1}}}
\@namedef{PY@tok@gi}{\def\PY@tc##1{\textcolor[rgb]{0.00,0.63,0.00}{##1}}}
\@namedef{PY@tok@gr}{\def\PY@tc##1{\textcolor[rgb]{1.00,0.00,0.00}{##1}}}
\@namedef{PY@tok@ge}{\let\PY@it=\textit}
\@namedef{PY@tok@gs}{\let\PY@bf=\textbf}
\@namedef{PY@tok@gp}{\let\PY@bf=\textbf\def\PY@tc##1{\textcolor[rgb]{0.00,0.00,0.50}{##1}}}
\@namedef{PY@tok@go}{\def\PY@tc##1{\textcolor[rgb]{0.53,0.53,0.53}{##1}}}
\@namedef{PY@tok@gt}{\def\PY@tc##1{\textcolor[rgb]{0.00,0.27,0.87}{##1}}}
\@namedef{PY@tok@err}{\def\PY@bc##1{{\setlength{\fboxsep}{\string -\fboxrule}\fcolorbox[rgb]{1.00,0.00,0.00}{1,1,1}{\strut ##1}}}}
\@namedef{PY@tok@kc}{\let\PY@bf=\textbf\def\PY@tc##1{\textcolor[rgb]{0.00,0.50,0.00}{##1}}}
\@namedef{PY@tok@kd}{\let\PY@bf=\textbf\def\PY@tc##1{\textcolor[rgb]{0.00,0.50,0.00}{##1}}}
\@namedef{PY@tok@kn}{\let\PY@bf=\textbf\def\PY@tc##1{\textcolor[rgb]{0.00,0.50,0.00}{##1}}}
\@namedef{PY@tok@kr}{\let\PY@bf=\textbf\def\PY@tc##1{\textcolor[rgb]{0.00,0.50,0.00}{##1}}}
\@namedef{PY@tok@bp}{\def\PY@tc##1{\textcolor[rgb]{0.00,0.50,0.00}{##1}}}
\@namedef{PY@tok@fm}{\def\PY@tc##1{\textcolor[rgb]{0.00,0.00,1.00}{##1}}}
\@namedef{PY@tok@vc}{\def\PY@tc##1{\textcolor[rgb]{0.10,0.09,0.49}{##1}}}
\@namedef{PY@tok@vg}{\def\PY@tc##1{\textcolor[rgb]{0.10,0.09,0.49}{##1}}}
\@namedef{PY@tok@vi}{\def\PY@tc##1{\textcolor[rgb]{0.10,0.09,0.49}{##1}}}
\@namedef{PY@tok@vm}{\def\PY@tc##1{\textcolor[rgb]{0.10,0.09,0.49}{##1}}}
\@namedef{PY@tok@sa}{\def\PY@tc##1{\textcolor[rgb]{0.73,0.13,0.13}{##1}}}
\@namedef{PY@tok@sb}{\def\PY@tc##1{\textcolor[rgb]{0.73,0.13,0.13}{##1}}}
\@namedef{PY@tok@sc}{\def\PY@tc##1{\textcolor[rgb]{0.73,0.13,0.13}{##1}}}
\@namedef{PY@tok@dl}{\def\PY@tc##1{\textcolor[rgb]{0.73,0.13,0.13}{##1}}}
\@namedef{PY@tok@s2}{\def\PY@tc##1{\textcolor[rgb]{0.73,0.13,0.13}{##1}}}
\@namedef{PY@tok@sh}{\def\PY@tc##1{\textcolor[rgb]{0.73,0.13,0.13}{##1}}}
\@namedef{PY@tok@s1}{\def\PY@tc##1{\textcolor[rgb]{0.73,0.13,0.13}{##1}}}
\@namedef{PY@tok@mb}{\def\PY@tc##1{\textcolor[rgb]{0.40,0.40,0.40}{##1}}}
\@namedef{PY@tok@mf}{\def\PY@tc##1{\textcolor[rgb]{0.40,0.40,0.40}{##1}}}
\@namedef{PY@tok@mh}{\def\PY@tc##1{\textcolor[rgb]{0.40,0.40,0.40}{##1}}}
\@namedef{PY@tok@mi}{\def\PY@tc##1{\textcolor[rgb]{0.40,0.40,0.40}{##1}}}
\@namedef{PY@tok@il}{\def\PY@tc##1{\textcolor[rgb]{0.40,0.40,0.40}{##1}}}
\@namedef{PY@tok@mo}{\def\PY@tc##1{\textcolor[rgb]{0.40,0.40,0.40}{##1}}}
\@namedef{PY@tok@ch}{\let\PY@it=\textit\def\PY@tc##1{\textcolor[rgb]{0.25,0.50,0.50}{##1}}}
\@namedef{PY@tok@cm}{\let\PY@it=\textit\def\PY@tc##1{\textcolor[rgb]{0.25,0.50,0.50}{##1}}}
\@namedef{PY@tok@cpf}{\let\PY@it=\textit\def\PY@tc##1{\textcolor[rgb]{0.25,0.50,0.50}{##1}}}
\@namedef{PY@tok@c1}{\let\PY@it=\textit\def\PY@tc##1{\textcolor[rgb]{0.25,0.50,0.50}{##1}}}
\@namedef{PY@tok@cs}{\let\PY@it=\textit\def\PY@tc##1{\textcolor[rgb]{0.25,0.50,0.50}{##1}}}

\def\PYZbs{\char`\\}
\def\PYZus{\char`\_}
\def\PYZob{\char`\{}
\def\PYZcb{\char`\}}
\def\PYZca{\char`\^}
\def\PYZam{\char`\&}
\def\PYZlt{\char`\<}
\def\PYZgt{\char`\>}
\def\PYZsh{\char`\#}
\def\PYZpc{\char`\%}
\def\PYZdl{\char`\$}
\def\PYZhy{\char`\-}
\def\PYZsq{\char`\'}
\def\PYZdq{\char`\"}
\def\PYZti{\char`\~}
% for compatibility with earlier versions
\def\PYZat{@}
\def\PYZlb{[}
\def\PYZrb{]}
\makeatother


    % For linebreaks inside Verbatim environment from package fancyvrb. 
    \makeatletter
        \newbox\Wrappedcontinuationbox 
        \newbox\Wrappedvisiblespacebox 
        \newcommand*\Wrappedvisiblespace {\textcolor{red}{\textvisiblespace}} 
        \newcommand*\Wrappedcontinuationsymbol {\textcolor{red}{\llap{\tiny$\m@th\hookrightarrow$}}} 
        \newcommand*\Wrappedcontinuationindent {3ex } 
        \newcommand*\Wrappedafterbreak {\kern\Wrappedcontinuationindent\copy\Wrappedcontinuationbox} 
        % Take advantage of the already applied Pygments mark-up to insert 
        % potential linebreaks for TeX processing. 
        %        {, <, #, %, $, ' and ": go to next line. 
        %        _, }, ^, &, >, - and ~: stay at end of broken line. 
        % Use of \textquotesingle for straight quote. 
        \newcommand*\Wrappedbreaksatspecials {% 
            \def\PYGZus{\discretionary{\char`\_}{\Wrappedafterbreak}{\char`\_}}% 
            \def\PYGZob{\discretionary{}{\Wrappedafterbreak\char`\{}{\char`\{}}% 
            \def\PYGZcb{\discretionary{\char`\}}{\Wrappedafterbreak}{\char`\}}}% 
            \def\PYGZca{\discretionary{\char`\^}{\Wrappedafterbreak}{\char`\^}}% 
            \def\PYGZam{\discretionary{\char`\&}{\Wrappedafterbreak}{\char`\&}}% 
            \def\PYGZlt{\discretionary{}{\Wrappedafterbreak\char`\<}{\char`\<}}% 
            \def\PYGZgt{\discretionary{\char`\>}{\Wrappedafterbreak}{\char`\>}}% 
            \def\PYGZsh{\discretionary{}{\Wrappedafterbreak\char`\#}{\char`\#}}% 
            \def\PYGZpc{\discretionary{}{\Wrappedafterbreak\char`\%}{\char`\%}}% 
            \def\PYGZdl{\discretionary{}{\Wrappedafterbreak\char`\$}{\char`\$}}% 
            \def\PYGZhy{\discretionary{\char`\-}{\Wrappedafterbreak}{\char`\-}}% 
            \def\PYGZsq{\discretionary{}{\Wrappedafterbreak\textquotesingle}{\textquotesingle}}% 
            \def\PYGZdq{\discretionary{}{\Wrappedafterbreak\char`\"}{\char`\"}}% 
            \def\PYGZti{\discretionary{\char`\~}{\Wrappedafterbreak}{\char`\~}}% 
        } 
        % Some characters . , ; ? ! / are not pygmentized. 
        % This macro makes them "active" and they will insert potential linebreaks 
        \newcommand*\Wrappedbreaksatpunct {% 
            \lccode`\~`\.\lowercase{\def~}{\discretionary{\hbox{\char`\.}}{\Wrappedafterbreak}{\hbox{\char`\.}}}% 
            \lccode`\~`\,\lowercase{\def~}{\discretionary{\hbox{\char`\,}}{\Wrappedafterbreak}{\hbox{\char`\,}}}% 
            \lccode`\~`\;\lowercase{\def~}{\discretionary{\hbox{\char`\;}}{\Wrappedafterbreak}{\hbox{\char`\;}}}% 
            \lccode`\~`\:\lowercase{\def~}{\discretionary{\hbox{\char`\:}}{\Wrappedafterbreak}{\hbox{\char`\:}}}% 
            \lccode`\~`\?\lowercase{\def~}{\discretionary{\hbox{\char`\?}}{\Wrappedafterbreak}{\hbox{\char`\?}}}% 
            \lccode`\~`\!\lowercase{\def~}{\discretionary{\hbox{\char`\!}}{\Wrappedafterbreak}{\hbox{\char`\!}}}% 
            \lccode`\~`\/\lowercase{\def~}{\discretionary{\hbox{\char`\/}}{\Wrappedafterbreak}{\hbox{\char`\/}}}% 
            \catcode`\.\active
            \catcode`\,\active 
            \catcode`\;\active
            \catcode`\:\active
            \catcode`\?\active
            \catcode`\!\active
            \catcode`\/\active 
            \lccode`\~`\~ 	
        }
    \makeatother

    \let\OriginalVerbatim=\Verbatim
    \makeatletter
    \renewcommand{\Verbatim}[1][1]{%
        %\parskip\z@skip
        \sbox\Wrappedcontinuationbox {\Wrappedcontinuationsymbol}%
        \sbox\Wrappedvisiblespacebox {\FV@SetupFont\Wrappedvisiblespace}%
        \def\FancyVerbFormatLine ##1{\hsize\linewidth
            \vtop{\raggedright\hyphenpenalty\z@\exhyphenpenalty\z@
                \doublehyphendemerits\z@\finalhyphendemerits\z@
                \strut ##1\strut}%
        }%
        % If the linebreak is at a space, the latter will be displayed as visible
        % space at end of first line, and a continuation symbol starts next line.
        % Stretch/shrink are however usually zero for typewriter font.
        \def\FV@Space {%
            \nobreak\hskip\z@ plus\fontdimen3\font minus\fontdimen4\font
            \discretionary{\copy\Wrappedvisiblespacebox}{\Wrappedafterbreak}
            {\kern\fontdimen2\font}%
        }%
        
        % Allow breaks at special characters using \PYG... macros.
        \Wrappedbreaksatspecials
        % Breaks at punctuation characters . , ; ? ! and / need catcode=\active 	
        \OriginalVerbatim[#1,codes*=\Wrappedbreaksatpunct]%
    }
    \makeatother

    % Exact colors from NB
    \definecolor{incolor}{HTML}{303F9F}
    \definecolor{outcolor}{HTML}{D84315}
    \definecolor{cellborder}{HTML}{CFCFCF}
    \definecolor{cellbackground}{HTML}{F7F7F7}
    
    % prompt
    \makeatletter
    \newcommand{\boxspacing}{\kern\kvtcb@left@rule\kern\kvtcb@boxsep}
    \makeatother
    \newcommand{\prompt}[4]{
        {\ttfamily\llap{{\color{#2}[#3]:\hspace{3pt}#4}}\vspace{-\baselineskip}}
    }
    

    
    % Prevent overflowing lines due to hard-to-break entities
    \sloppy 
    % Setup hyperref package
    \hypersetup{
      breaklinks=true,  % so long urls are correctly broken across lines
      colorlinks=true,
      urlcolor=urlcolor,
      linkcolor=linkcolor,
      citecolor=citecolor,
      }
    % Slightly bigger margins than the latex defaults
    
    \geometry{verbose,tmargin=1in,bmargin=1in,lmargin=1in,rmargin=1in}
    
    

\begin{document}
    
    \maketitle
    
\tableofcontents{}
    
\section{Введение}
Лабораторная работа по теме 2.2 "Биоматериалы и биомеханика ткани" курса М1.Б.4 "Современные проблемы биомедицинской и экологической инженерии" для подготовки магистров по направлению 12.04.04 "Биотехнические системы и технологии".

На лекциях были рассмотрены Ленгмюровские пленки.

Практическое занятие позволяет на практике освоить методы компьютерного зрения при работе с Ленгмюровскими пленками. 


\section{Цель}
Получить навыки применения практических данных в построении графического материала для визуализации научных данных и результатов расчетов для углубления компетенции представлять современную научную картину мира, выявлять естественнонаучную сущность проблемы, формулировать задачи, определять пути их решения и оценивать эффективность выбора и методов правовой защиты результатов интеллектуальной деятельности с учетом исследований, разработки и проектирования биотехнических систем и технологий.
\section{Теория пленок Ленгмюра-Блоджетта}    

\subsection{Обзор}    
В 1980-х годах был большой энтузиазм по поводу молекулярных сборок на основе пленок Ленгмюра-Блоджетт (LB). Были предложены концепции для следующего тысячелетия, такие как «Молекулярная электроника, в которой органические молекулы выполняют активную функцию в обработке информации, а также в передаче и хранении» ~\cite{c1,c2,c3}. Эти предложения вызвали много ожиданий. 

На текущий момент пленки LB служат в основном модельными поверхностями для изучения взаимодействий.
  
Как было показано в лекции,в исследованиях и производстве биоматериалов применяются, как обычные пленки LB, так и новые типы, полученные путем организации наночастиц на границе воздух-вода. 

Исчерпывающую информацию по LB можно найти в ~\cite{Handbook}, далее будут приведены лишь выборочные материалы из этой публикации.

\subsection{Что делает пленки LB привлекательными?}  
Привлекательными свойствами пленок Ленгмюра-Блоджетт являются внутренний контроль структуры внутреннего слоя вплоть до молекулярного уровня и точный контроль толщины полученной пленки.

Сложные желоба LB позволяют нам обрабатывать несколько материалов с различными функциональными возможностями и дают возможность настраивать архитектуру слоев в соответствии с требованиями желаемых молекулярно-инженерных органических тонкопленочных устройств. 

Более подробную информацию можно найти в книгах Гейнса ~\cite{c4} или Ульмана ~\cite{c5}; последний также представляет собой хорошее введение в инструменты анализа поверхности, которые обычно используются для исследования структуры монослоев и различных физических свойств.
Первым этапом процесса осаждения LB является формирование четко определенного монослоя на границе воздух-вода. Эти так называемые ленгмюровские монослои являются пленкой-предшественником в производстве LB. Процесс подготовки схематично показан на следующей
 поэтапной схеме процесса осаждения Ленгмюра-Блогетта. 
 
 Амфифил растворяют в органическом растворителе и затем распределяют на границе воздух-вода. Растворитель испаряется и остается монослой амфифила на границе воздух-вода. Монослоем на границе воздух-вода можно дополнительно управлять с помощью подвижного барьера, позволяющего контролировать площадь на молекулу.

    \begin{center}
    \adjustimage{max size={\linewidth}{\paperheight}}{28_1_a.jpg}
    \end{center}    

Монослоем на границе воздух-вода можно дополнительно управлять с помощью подвижного барьера, позволяющего контролировать площадь на молекулу

    \begin{center}
    \adjustimage{max size={\linewidth}{\paperheight}}{28_1_b.jpg}
    \end{center}    


Монослой Ленгмюра можно переносить движением вверх на гидрофильную поверхность

    \begin{center}
    \adjustimage{max size={\linewidth}{\paperheight}}{28_1_c.jpg}
    \end{center}    
    
и движением вниз на гидрофобную поверхность. Ванночка с двумя отсеками позволяет одновременно обрабатывать два разных материала

    \begin{center}
    \adjustimage{max size={\linewidth}{\paperheight}}{28_1_d.jpg}
    \end{center}    

 а запрограммированная последовательность погружения позволяет определять архитектуру слоя на молекулярном уровне.
 
     \begin{center}
    \adjustimage{max size={\linewidth}{\paperheight}}{28_1_e.jpg}
    \end{center}  
    
  Монослои на границе воздух-вода были тщательно изучены и обладают богатством фаз и структур. Они служат квазидвумерными модельными системами и, таким образом, привлекли значительное внимание исследователей. 
  
  Появление сложных инструментов анализа поверхности, таких как методы отражения и рассеяния рентгеновских лучей, вместе с новыми оптическими методами, такими как флуоресцентная микроскопия и микроскопия под углом Брюстера (BAM), предоставили подробную картину общих фазовых диаграмм, структуры и морфологии. 
  
  Рентгеновская дифракция скользящего падения выявила существование нескольких фаз, в которых алифатическая цепь наклонена относительно нормали к поверхности и в которых азимут наклона принимает четко определенное расположение относительно лежащего в основе порядка ориентации связи ~\cite{c6}. 
  
  Организация азимута наклона может распространяться на макроскопические размеры и, таким образом, проявляется на изображениях BAM в различных аспектах, таких как образование доменов с внутренней структурой. 
  
  Обзорные статьи Mohwald ~\cite{c7}, McConnel ~\cite{c8}, Knobler and Desai ~\cite{c9}, Riviere et al ~\cite{c10} и Knobler and Schwartz ~\cite{c11} являются прекрасными проводниками по огромному количеству публикаций, посвященных монослоям Ленгмюра.

\section{Модель}
Модель расчета повторяет эксперимент по исследованию физического поведения пептида KR-12 липидных пленках Ленгмюра-Блоджетта при его антимикробном действии против S.aureus и демонстрирует алгоритм определения толщины с помощью компьютерного зрения. ~\cite{Chanci}

В статье были представлены результаты, приведенные на следующем рисунке:

    \begin{center}
    \adjustimage{max size={\linewidth}{\paperheight}}{model.jpg}
    \end{center}    

Для проведения практической работы воспользуемся лабораторными фотографиями опубливанных там же ~\cite{Chanci}.
    
    
\section{Практическая работа}        
    \subsection{Подготовка окружения выполнения}
Для успешного выполнения кода на своем компьютере необходимо выполнить ряд подготовительных мероприятий.

\subsubsection{Установка Anaconda}
Anaconda - дистрибутив языков программирования Python и R, включающий набор популярных свободных библиотек, объединённых проблематиками науки о данных и машинного обучения.

Для скачивания пакета Anaconda надо перейти по ссылке и скачать дистрибутив для своей операционной системы:

~\url{https://www.anaconda.com/products/individual}

    \begin{center}
    \adjustimage{max size={0.9\linewidth}{0.9\paperheight}}{1.png}
    \end{center}

После установки необходимо запустить навигатор. В Windows и MacOS X его можно найти в списке приложений, в Linux его можно запустить  из консоли 
    \begin{Verbatim}[commandchars=\\\{\}]
./anaconda-navigator
    \end{Verbatim}
     
Окно навигатора показано на рисунке ниже.
    \begin{center}
    \adjustimage{max size={0.9\linewidth}{0.9\paperheight}}{2.png}
    \end{center}

\subsubsection{Установка зависимостей}
После установки Anaconda нам надо установить дополнительные библиотеки, для этого в консоли необходимо выполнить следующие команды. Их выполнение может занять значительное время, поэтому не стоит их прерывать из-за предположения, что они зависли. Если у вас возникнут сложности, поскольку со временем что то может измениться, то всегда можно обратиться с поиском в интернет и "загуглить" ошибку и способ ее устранения.

    \begin{Verbatim}[commandchars=\\\{\}]
conda update --all --yes
conda install -c conda-forge google-colab
conda install -c conda-forge opencv
conda install -c conda-forge pandoc
conda install -c conda-forge nbconvert
conda install -c conda-forge pyppeteer
    \end{Verbatim}


\subsection{Пошаговая инструкция для создания блокнота в Jupyter Notebook}
    \begin{tcolorbox}[breakable, size=fbox, boxrule=1pt, pad at break*=1mm,colback=cellbackground, colframe=cellborder]
\prompt{In}{incolor}{51}{\boxspacing}
\begin{Verbatim}[commandchars=\\\{\}]
\PY{k+kn}{import} \PY{n+nn}{numpy} \PY{k}{as} \PY{n+nn}{np}
\PY{k+kn}{import} \PY{n+nn}{matplotlib}\PY{n+nn}{.}\PY{n+nn}{pyplot} \PY{k}{as} \PY{n+nn}{plt}
\PY{k+kn}{import} \PY{n+nn}{pandas} \PY{k}{as} \PY{n+nn}{pd}
\PY{k+kn}{import} \PY{n+nn}{cv2}                        \PY{c+c1}{\PYZsh{}Библиотека компьютерного зрения OpenCV}
\end{Verbatim}
\end{tcolorbox}

    ** Изображение из https://doi.org/10.1016/j.bbamem.2021.183779 **

    \begin{tcolorbox}[breakable, size=fbox, boxrule=1pt, pad at break*=1mm,colback=cellbackground, colframe=cellborder]
\prompt{In}{incolor}{52}{\boxspacing}
\begin{Verbatim}[commandchars=\\\{\}]
\PY{c+c1}{\PYZsh{}Загружаем изображение}

\PY{n}{img} \PY{o}{=} \PY{n}{cv2}\PY{o}{.}\PY{n}{imread}\PY{p}{(}\PY{l+s+s1}{\PYZsq{}}\PY{l+s+s1}{images/PGPGCL\PYZus{}H2O\PYZus{}004.jpg}\PY{l+s+s1}{\PYZsq{}}\PY{p}{)}
\PY{n}{img} \PY{o}{=} \PY{n}{cv2}\PY{o}{.}\PY{n}{cvtColor}\PY{p}{(}\PY{n}{img}\PY{p}{,}\PY{n}{cv2}\PY{o}{.}\PY{n}{COLOR\PYZus{}BGR2RGB}\PY{p}{)}

\PY{c+c1}{\PYZsh{}Обрезаем изображение}

\PY{n}{cv2}\PY{o}{.}\PY{n}{rectangle}\PY{p}{(}\PY{n}{img}\PY{p}{,}\PY{p}{(}\PY{l+m+mi}{30}\PY{p}{,}\PY{l+m+mi}{30}\PY{p}{)}\PY{p}{,}\PY{p}{(}\PY{l+m+mi}{800}\PY{p}{,}\PY{l+m+mi}{800}\PY{p}{)}\PY{p}{,}\PY{p}{(}\PY{l+m+mi}{255}\PY{p}{,}\PY{l+m+mi}{0}\PY{p}{,}\PY{l+m+mi}{0}\PY{p}{)}\PY{p}{,}\PY{l+m+mi}{5}\PY{p}{)}\PY{p}{;}       \PY{c+c1}{\PYZsh{}Красный квадрат }
\PY{n}{cv2}\PY{o}{.}\PY{n}{rectangle}\PY{p}{(}\PY{n}{img}\PY{p}{,}\PY{p}{(}\PY{l+m+mi}{1000}\PY{p}{,}\PY{l+m+mi}{28}\PY{p}{)}\PY{p}{,}\PY{p}{(}\PY{l+m+mi}{1000}\PY{p}{,}\PY{l+m+mi}{1048}\PY{p}{)}\PY{p}{,}\PY{p}{(}\PY{l+m+mi}{255}\PY{p}{,}\PY{l+m+mi}{0}\PY{p}{,}\PY{l+m+mi}{0}\PY{p}{)}\PY{p}{,}\PY{l+m+mi}{5}\PY{p}{)}\PY{p}{;}   \PY{c+c1}{\PYZsh{}Красная линия}
\PY{n}{plt}\PY{o}{.}\PY{n}{figure}\PY{p}{(}\PY{n}{figsize}\PY{o}{=}\PY{p}{(}\PY{l+m+mi}{10}\PY{p}{,}\PY{l+m+mi}{10}\PY{p}{)}\PY{p}{)}
\PY{n}{plt}\PY{o}{.}\PY{n}{imshow}\PY{p}{(}\PY{n}{img}\PY{p}{)}
\end{Verbatim}
\end{tcolorbox}

            \begin{tcolorbox}[breakable, size=fbox, boxrule=.5pt, pad at break*=1mm, opacityfill=0]
\prompt{Out}{outcolor}{52}{\boxspacing}
\begin{Verbatim}[commandchars=\\\{\}]
<matplotlib.image.AxesImage at 0x7fd61a89ff70>
\end{Verbatim}
\end{tcolorbox}
        
    \begin{center}
    \adjustimage{max size={0.9\linewidth}{0.9\paperheight}}{output_2_1.png}
    \end{center}
    { \hspace*{\fill} \\}
    
    \begin{tcolorbox}[breakable, size=fbox, boxrule=1pt, pad at break*=1mm,colback=cellbackground, colframe=cellborder]
\prompt{In}{incolor}{53}{\boxspacing}
\begin{Verbatim}[commandchars=\\\{\}]
\PY{c+c1}{\PYZsh{}Переименуем красный квадрат и красную линию}

\PY{n}{bilayerImg} \PY{o}{=} \PY{n}{img}\PY{p}{[}\PY{l+m+mi}{30}\PY{p}{:}\PY{l+m+mi}{800}\PY{p}{,}\PY{l+m+mi}{30}\PY{p}{:}\PY{l+m+mi}{800}\PY{p}{]}      \PY{c+c1}{\PYZsh{}Красный квадрат (двухслойный/bilayer)}
\PY{n}{scaleImg} \PY{o}{=} \PY{n}{img}\PY{p}{[}\PY{l+m+mi}{28}\PY{p}{:}\PY{l+m+mi}{1048}\PY{p}{,}\PY{l+m+mi}{1000}\PY{p}{]}         \PY{c+c1}{\PYZsh{}Красная линия (шкала высот)}
\PY{n}{plt}\PY{o}{.}\PY{n}{figure}\PY{p}{(}\PY{n}{figsize}\PY{o}{=}\PY{p}{(}\PY{l+m+mi}{10}\PY{p}{,}\PY{l+m+mi}{10}\PY{p}{)}\PY{p}{)}
\PY{n}{plt}\PY{o}{.}\PY{n}{imshow}\PY{p}{(}\PY{n}{bilayerImg}\PY{p}{)}
\end{Verbatim}
\end{tcolorbox}

            \begin{tcolorbox}[breakable, size=fbox, boxrule=.5pt, pad at break*=1mm, opacityfill=0]
\prompt{Out}{outcolor}{53}{\boxspacing}
\begin{Verbatim}[commandchars=\\\{\}]
<matplotlib.image.AxesImage at 0x7fd61aac7b50>
\end{Verbatim}
\end{tcolorbox}
        
    \begin{center}
    \adjustimage{max size={0.9\linewidth}{0.9\paperheight}}{output_3_1.png}
    \end{center}
    { \hspace*{\fill} \\}
    
    \begin{tcolorbox}[breakable, size=fbox, boxrule=1pt, pad at break*=1mm,colback=cellbackground, colframe=cellborder]
\prompt{In}{incolor}{54}{\boxspacing}
\begin{Verbatim}[commandchars=\\\{\}]
\PY{n}{bilayerImg}\PY{o}{.}\PY{n}{shape} \PY{c+c1}{\PYZsh{}Получить параметры красного квадрата в пикселях}
\end{Verbatim}
\end{tcolorbox}

            \begin{tcolorbox}[breakable, size=fbox, boxrule=.5pt, pad at break*=1mm, opacityfill=0]
\prompt{Out}{outcolor}{54}{\boxspacing}
\begin{Verbatim}[commandchars=\\\{\}]
(770, 770, 3)
\end{Verbatim}
\end{tcolorbox}
        
    \begin{tcolorbox}[breakable, size=fbox, boxrule=1pt, pad at break*=1mm,colback=cellbackground, colframe=cellborder]
\prompt{In}{incolor}{55}{\boxspacing}
\begin{Verbatim}[commandchars=\\\{\}]
\PY{c+c1}{\PYZsh{}Средний цвет RGB красного квадрата}

\PY{n}{promRGB} \PY{o}{=} \PY{n}{np}\PY{o}{.}\PY{n}{mean}\PY{p}{(}\PY{n}{bilayerImg}\PY{p}{,} \PY{n}{axis}\PY{o}{=}\PY{p}{(}\PY{l+m+mi}{0}\PY{p}{,}\PY{l+m+mi}{1}\PY{p}{)}\PY{p}{)}
\PY{n}{promRGB} \PY{o}{=} \PY{p}{[}\PY{n+nb}{int}\PY{p}{(}\PY{n}{i}\PY{p}{)} \PY{k}{for} \PY{n}{i} \PY{o+ow}{in} \PY{n}{promRGB}\PY{p}{]}
\PY{n}{promRGB}
\end{Verbatim}
\end{tcolorbox}

            \begin{tcolorbox}[breakable, size=fbox, boxrule=.5pt, pad at break*=1mm, opacityfill=0]
\prompt{Out}{outcolor}{55}{\boxspacing}
\begin{Verbatim}[commandchars=\\\{\}]
[132, 66, 2]
\end{Verbatim}
\end{tcolorbox}
        
    \begin{tcolorbox}[breakable, size=fbox, boxrule=1pt, pad at break*=1mm,colback=cellbackground, colframe=cellborder]
\prompt{In}{incolor}{56}{\boxspacing}
\begin{Verbatim}[commandchars=\\\{\}]
\PY{c+c1}{\PYZsh{}Проверьте значения красной линии для i в scaleImg:}
\PY{k}{for} \PY{n}{i} \PY{o+ow}{in} \PY{n}{scaleImg}\PY{p}{:}
    \PY{n+nb}{print}\PY{p}{(}\PY{n}{i}\PY{p}{)}
\end{Verbatim}
\end{tcolorbox}

    \begin{Verbatim}[commandchars=\\\{\}]
[255   0   0]
...
[255   0   0]
    \end{Verbatim}

    \begin{tcolorbox}[breakable, size=fbox, boxrule=1pt, pad at break*=1mm,colback=cellbackground, colframe=cellborder]
\prompt{In}{incolor}{57}{\boxspacing}
\begin{Verbatim}[commandchars=\\\{\}]
\PY{c+c1}{\PYZsh{}Сравните цвет RGB каждого пикселя в красном квадрате с цветом RGB каждого пикселя в красной линии.}

\PY{n}{c} \PY{o}{=} \PY{l+m+mi}{0}
\PY{k}{for} \PY{n}{i} \PY{o+ow}{in} \PY{n}{scaleImg}\PY{p}{:}
    \PY{n}{c} \PY{o}{+}\PY{o}{=} \PY{l+m+mi}{1}
    \PY{k}{if} \PY{n}{np}\PY{o}{.}\PY{n}{sum}\PY{p}{(}\PY{n+nb}{abs}\PY{p}{(}\PY{n}{i} \PY{o}{\PYZhy{}} \PY{n}{np}\PY{o}{.}\PY{n}{array}\PY{p}{(}\PY{n}{promRGB}\PY{p}{)}\PY{p}{)}\PY{p}{)} \PY{o}{\PYZlt{}} \PY{l+m+mi}{7}\PY{p}{:}
        \PY{n+nb}{print}\PY{p}{(}\PY{l+s+s1}{\PYZsq{}}\PY{l+s+s1}{Match}\PY{l+s+s1}{\PYZsq{}}\PY{p}{,}\PY{n}{c}\PY{p}{)}                        \PY{c+c1}{\PYZsh{}Сопоставьте с более похожим цветом по шкале высоты}
\end{Verbatim}
\end{tcolorbox}

    467 индекс средней высоты

    \begin{tcolorbox}[breakable, size=fbox, boxrule=1pt, pad at break*=1mm,colback=cellbackground, colframe=cellborder]
\prompt{In}{incolor}{58}{\boxspacing}
\begin{Verbatim}[commandchars=\\\{\}]
\PY{c+c1}{\PYZsh{}Средняя высота бислоя}

\PY{n}{Bilayer\PYZus{}h} \PY{o}{=} \PY{l+m+mf}{3.2} \PY{o}{+} \PY{l+m+mf}{3.8}  \PY{c+c1}{\PYZsh{}Высота микрофотографии АСМ}
\PY{n}{Match} \PY{o}{=} \PY{l+m+mi}{467}            \PY{c+c1}{\PYZsh{}Индекс средней высоты}
\PY{n}{Scale\PYZus{}pixls} \PY{o}{=} \PY{l+m+mi}{1020}     \PY{c+c1}{\PYZsh{}Количество пикселей в красной линии }
\PY{n}{Top\PYZus{}h} \PY{o}{=} \PY{l+m+mf}{3.2}            \PY{c+c1}{\PYZsh{}Самая высокая точка на микрофотографии АСМ}

\PY{n}{h} \PY{o}{=} \PY{p}{(}\PY{n}{Match}\PY{o}{*}\PY{n}{Bilayer\PYZus{}h}\PY{p}{)}\PY{o}{/}\PY{n}{Scale\PYZus{}pixls}
\PY{n}{Real\PYZus{}h} \PY{o}{=} \PY{n}{Top\PYZus{}h}\PY{o}{\PYZhy{}}\PY{n}{h}
\PY{n+nb}{print}\PY{p}{(}\PY{n}{Real\PYZus{}h}\PY{p}{)}
\end{Verbatim}
\end{tcolorbox}

    \begin{Verbatim}[commandchars=\\\{\}]
-0.004901960784313708
    \end{Verbatim}

    В среднем липидный бислой был однородным.

    Изображение из https://doi.org/10.1016/j.bbamem.2021.183779

    \begin{tcolorbox}[breakable, size=fbox, boxrule=1pt, pad at break*=1mm,colback=cellbackground, colframe=cellborder]
\prompt{In}{incolor}{59}{\boxspacing}
\begin{Verbatim}[commandchars=\\\{\}]
\PY{c+c1}{\PYZsh{}Загружаем изображение}

\PY{n}{img2} \PY{o}{=} \PY{n}{cv2}\PY{o}{.}\PY{n}{imread}\PY{p}{(}\PY{l+s+s1}{\PYZsq{}}\PY{l+s+s1}{images/PGPGCL\PYZus{}H2O\PYZus{}AMP\PYZus{}007\PYZus{}KR.jpg}\PY{l+s+s1}{\PYZsq{}}\PY{p}{)}
\PY{n}{img2} \PY{o}{=} \PY{n}{cv2}\PY{o}{.}\PY{n}{cvtColor}\PY{p}{(}\PY{n}{img2}\PY{p}{,}\PY{n}{cv2}\PY{o}{.}\PY{n}{COLOR\PYZus{}BGR2RGB}\PY{p}{)}

\PY{c+c1}{\PYZsh{}Обрезаем изображение}

\PY{n}{bilayerImg} \PY{o}{=} \PY{n}{img2}\PY{p}{[}\PY{l+m+mi}{30}\PY{p}{:}\PY{l+m+mi}{800}\PY{p}{,}\PY{l+m+mi}{30}\PY{p}{:}\PY{l+m+mi}{800}\PY{p}{]}
\PY{n}{scaleImg} \PY{o}{=} \PY{n}{img2}\PY{p}{[}\PY{l+m+mi}{28}\PY{p}{:}\PY{l+m+mi}{1048}\PY{p}{,}\PY{l+m+mi}{1000}\PY{p}{]}
\PY{n}{plt}\PY{o}{.}\PY{n}{figure}\PY{p}{(}\PY{n}{figsize}\PY{o}{=}\PY{p}{(}\PY{l+m+mi}{10}\PY{p}{,}\PY{l+m+mi}{10}\PY{p}{)}\PY{p}{)}
\PY{n}{plt}\PY{o}{.}\PY{n}{imshow}\PY{p}{(}\PY{n}{bilayerImg}\PY{p}{)}
\end{Verbatim}
\end{tcolorbox}

            \begin{tcolorbox}[breakable, size=fbox, boxrule=.5pt, pad at break*=1mm, opacityfill=0]
\prompt{Out}{outcolor}{59}{\boxspacing}
\begin{Verbatim}[commandchars=\\\{\}]
<matplotlib.image.AxesImage at 0x7fd61a7e4460>
\end{Verbatim}
\end{tcolorbox}
        
    \begin{center}
    \adjustimage{max size={0.9\linewidth}{0.9\paperheight}}{output_12_1.png}
    \end{center}
    { \hspace*{\fill} \\}
    
    \begin{tcolorbox}[breakable, size=fbox, boxrule=1pt, pad at break*=1mm,colback=cellbackground, colframe=cellborder]
\prompt{In}{incolor}{60}{\boxspacing}
\begin{Verbatim}[commandchars=\\\{\}]
\PY{c+c1}{\PYZsh{} Создайте график гистограммы с тремя линиями, по одной для каждого цвета.}

\PY{n}{colors} \PY{o}{=} \PY{p}{(}\PY{l+s+s2}{\PYZdq{}}\PY{l+s+s2}{r}\PY{l+s+s2}{\PYZdq{}}\PY{p}{,} \PY{l+s+s2}{\PYZdq{}}\PY{l+s+s2}{g}\PY{l+s+s2}{\PYZdq{}}\PY{p}{,} \PY{l+s+s2}{\PYZdq{}}\PY{l+s+s2}{b}\PY{l+s+s2}{\PYZdq{}}\PY{p}{)}
\PY{n}{channel\PYZus{}ids} \PY{o}{=} \PY{p}{(}\PY{l+m+mi}{0}\PY{p}{,} \PY{l+m+mi}{1}\PY{p}{,} \PY{l+m+mi}{2}\PY{p}{)}

\PY{n}{plt}\PY{o}{.}\PY{n}{xlim}\PY{p}{(}\PY{p}{[}\PY{l+m+mi}{0}\PY{p}{,} \PY{l+m+mi}{256}\PY{p}{]}\PY{p}{)}
\PY{k}{for} \PY{n}{channel\PYZus{}id}\PY{p}{,} \PY{n}{c} \PY{o+ow}{in} \PY{n+nb}{zip}\PY{p}{(}\PY{n}{channel\PYZus{}ids}\PY{p}{,} \PY{n}{colors}\PY{p}{)}\PY{p}{:}
    \PY{n}{histogram}\PY{p}{,} \PY{n}{bin\PYZus{}edges} \PY{o}{=} \PY{n}{np}\PY{o}{.}\PY{n}{histogram}\PY{p}{(}
        \PY{n}{bilayerImg}\PY{p}{[}\PY{p}{:}\PY{p}{,} \PY{p}{:}\PY{p}{,} \PY{n}{channel\PYZus{}id}\PY{p}{]}\PY{p}{,} \PY{n}{bins}\PY{o}{=}\PY{l+m+mi}{256}\PY{p}{,} \PY{n+nb}{range}\PY{o}{=}\PY{p}{(}\PY{l+m+mi}{5}\PY{p}{,} \PY{l+m+mi}{256}\PY{p}{)}
    \PY{p}{)}
    \PY{n}{plt}\PY{o}{.}\PY{n}{plot}\PY{p}{(}\PY{n}{bin\PYZus{}edges}\PY{p}{[}\PY{l+m+mi}{0}\PY{p}{:}\PY{o}{\PYZhy{}}\PY{l+m+mi}{1}\PY{p}{]}\PY{p}{,} \PY{n}{histogram}\PY{p}{,} \PY{n}{color}\PY{o}{=}\PY{n}{c}\PY{p}{)}

\PY{n}{plt}\PY{o}{.}\PY{n}{xlabel}\PY{p}{(}\PY{l+s+s2}{\PYZdq{}}\PY{l+s+s2}{Цвет}\PY{l+s+s2}{\PYZdq{}}\PY{p}{)}
\PY{n}{plt}\PY{o}{.}\PY{n}{ylabel}\PY{p}{(}\PY{l+s+s2}{\PYZdq{}}\PY{l+s+s2}{Пиксель}\PY{l+s+s2}{\PYZdq{}}\PY{p}{)}

\PY{n}{plt}\PY{o}{.}\PY{n}{show}\PY{p}{(}\PY{p}{)}
\end{Verbatim}
\end{tcolorbox}

    \begin{center}
    \adjustimage{max size={0.9\linewidth}{0.9\paperheight}}{output_13_0.png}
    \end{center}
    { \hspace*{\fill} \\}
    
    \begin{tcolorbox}[breakable, size=fbox, boxrule=1pt, pad at break*=1mm,colback=cellbackground, colframe=cellborder]
\prompt{In}{incolor}{61}{\boxspacing}
\begin{Verbatim}[commandchars=\\\{\}]
\PY{c+c1}{\PYZsh{}Средний цвет RGB бислоя после 30 минут взаимодействия с КР\PYZhy{}12 AMP}

\PY{n}{promRGB} \PY{o}{=} \PY{n}{np}\PY{o}{.}\PY{n}{mean}\PY{p}{(}\PY{n}{bilayerImg}\PY{p}{,} \PY{n}{axis}\PY{o}{=}\PY{p}{(}\PY{l+m+mi}{0}\PY{p}{,}\PY{l+m+mi}{1}\PY{p}{)}\PY{p}{)}
\PY{n}{promRGB} \PY{o}{=} \PY{p}{[}\PY{n+nb}{int}\PY{p}{(}\PY{n}{i}\PY{p}{)} \PY{k}{for} \PY{n}{i} \PY{o+ow}{in} \PY{n}{promRGB}\PY{p}{]}
\PY{n}{promRGB}
\end{Verbatim}
\end{tcolorbox}

            \begin{tcolorbox}[breakable, size=fbox, boxrule=.5pt, pad at break*=1mm, opacityfill=0]
\prompt{Out}{outcolor}{61}{\boxspacing}
\begin{Verbatim}[commandchars=\\\{\}]
[86, 16, 5]
\end{Verbatim}
\end{tcolorbox}
        
    \begin{tcolorbox}[breakable, size=fbox, boxrule=1pt, pad at break*=1mm,colback=cellbackground, colframe=cellborder]
\prompt{In}{incolor}{62}{\boxspacing}
\begin{Verbatim}[commandchars=\\\{\}]
\PY{c+c1}{\PYZsh{}Сравните цвет RGB каждого пикселя в двухслойном изображении с цветом RGB каждого пикселя на шкале высоты.}

\PY{n}{c} \PY{o}{=} \PY{l+m+mi}{0}
\PY{n}{meann} \PY{o}{=} \PY{p}{[}\PY{p}{]}
\PY{k}{for} \PY{n}{i} \PY{o+ow}{in} \PY{n}{scaleImg}\PY{p}{:}
    \PY{n}{c} \PY{o}{+}\PY{o}{=} \PY{l+m+mi}{1}
    \PY{k}{if} \PY{n}{np}\PY{o}{.}\PY{n}{sum}\PY{p}{(}\PY{n+nb}{abs}\PY{p}{(}\PY{n}{i} \PY{o}{\PYZhy{}} \PY{n}{np}\PY{o}{.}\PY{n}{array}\PY{p}{(}\PY{n}{promRGB}\PY{p}{)}\PY{p}{)}\PY{p}{)} \PY{o}{\PYZlt{}} \PY{l+m+mi}{22}\PY{p}{:}
      \PY{n+nb}{print}\PY{p}{(}\PY{l+s+s1}{\PYZsq{}}\PY{l+s+s1}{Match}\PY{l+s+s1}{\PYZsq{}}\PY{p}{,}\PY{n}{c}\PY{p}{)}                            \PY{c+c1}{\PYZsh{}Сопоставьте с более похожим цветом на шкале высот}
      \PY{n}{meann}\PY{o}{.}\PY{n}{append}\PY{p}{(}\PY{n}{c}\PY{p}{)}
\end{Verbatim}
\end{tcolorbox}

    \begin{Verbatim}[commandchars=\\\{\}]
Match 581
Match 582
Match 583
Match 584
Match 585
Match 586
Match 587
Match 588
Match 589
Match 590
Match 591
Match 592
Match 593
Match 594
Match 595
Match 596
Match 632
Match 633
Match 634
Match 635
    \end{Verbatim}

    \begin{tcolorbox}[breakable, size=fbox, boxrule=1pt, pad at break*=1mm,colback=cellbackground, colframe=cellborder]
\prompt{In}{incolor}{63}{\boxspacing}
\begin{Verbatim}[commandchars=\\\{\}]
\PY{n}{np}\PY{o}{.}\PY{n}{mean}\PY{p}{(}\PY{n}{meann}\PY{p}{)} \PY{c+c1}{\PYZsh{}Индекс средней высоты}
\end{Verbatim}
\end{tcolorbox}

            \begin{tcolorbox}[breakable, size=fbox, boxrule=.5pt, pad at break*=1mm, opacityfill=0]
\prompt{Out}{outcolor}{63}{\boxspacing}
\begin{Verbatim}[commandchars=\\\{\}]
597.5
\end{Verbatim}
\end{tcolorbox}
        
    \begin{tcolorbox}[breakable, size=fbox, boxrule=1pt, pad at break*=1mm,colback=cellbackground, colframe=cellborder]
\prompt{In}{incolor}{64}{\boxspacing}
\begin{Verbatim}[commandchars=\\\{\}]
\PY{c+c1}{\PYZsh{}Средняя высота бислоя через 30 мин взаимодействия с пептидом}

\PY{n}{Bilayer\PYZus{}h} \PY{o}{=} \PY{l+m+mf}{14.6} \PY{o}{+} \PY{l+m+mf}{9.7}  \PY{c+c1}{\PYZsh{}Высота микрофотографии АСМ}
\PY{n}{Match} \PY{o}{=} \PY{l+m+mi}{597}             \PY{c+c1}{\PYZsh{}Индекс средней высоты}
\PY{n}{Scale\PYZus{}pixls} \PY{o}{=} \PY{l+m+mi}{1020}      \PY{c+c1}{\PYZsh{}Количество пикселей в красной линии }
\PY{n}{Top\PYZus{}h} \PY{o}{=} \PY{l+m+mf}{14.6}            \PY{c+c1}{\PYZsh{}Самая высокая точка на микрофотографии АСМ}

\PY{n}{h} \PY{o}{=} \PY{p}{(}\PY{n}{Match}\PY{o}{*}\PY{n}{Bilayer\PYZus{}h}\PY{p}{)}\PY{o}{/}\PY{n}{Scale\PYZus{}pixls}
\PY{n}{Real\PYZus{}h} \PY{o}{=} \PY{n}{Top\PYZus{}h}\PY{o}{\PYZhy{}}\PY{n}{h}
\PY{n+nb}{print}\PY{p}{(}\PY{n}{Real\PYZus{}h}\PY{p}{)}
\end{Verbatim}
\end{tcolorbox}

    \begin{Verbatim}[commandchars=\\\{\}]
0.377352941176472
    \end{Verbatim}

    \begin{tcolorbox}[breakable, size=fbox, boxrule=1pt, pad at break*=1mm,colback=cellbackground, colframe=cellborder]
\prompt{In}{incolor}{65}{\boxspacing}
\begin{Verbatim}[commandchars=\\\{\}]
\PY{c+c1}{\PYZsh{}Процесс сегментации}

\PY{n}{pixel\PYZus{}values} \PY{o}{=} \PY{n}{bilayerImg}\PY{o}{.}\PY{n}{reshape}\PY{p}{(}\PY{p}{(}\PY{o}{\PYZhy{}}\PY{l+m+mi}{1}\PY{p}{,} \PY{l+m+mi}{3}\PY{p}{)}\PY{p}{)} \PY{c+c1}{\PYZsh{}Изменить изображение}

\PY{n}{pixel\PYZus{}values} \PY{o}{=} \PY{n}{np}\PY{o}{.}\PY{n}{float32}\PY{p}{(}\PY{n}{pixel\PYZus{}values}\PY{p}{)}    \PY{c+c1}{\PYZsh{}Конвертируем в float}
\PY{c+c1}{\PYZsh{}pixel\PYZus{}values.shape}
\PY{c+c1}{\PYZsh{}pixel\PYZus{}values}

\PY{c+c1}{\PYZsh{}Критерий сегментации}
\PY{n}{criteria} \PY{o}{=} \PY{p}{(}\PY{n}{cv2}\PY{o}{.}\PY{n}{TERM\PYZus{}CRITERIA\PYZus{}EPS} \PY{o}{+} \PY{n}{cv2}\PY{o}{.}\PY{n}{TERM\PYZus{}CRITERIA\PYZus{}MAX\PYZus{}ITER}\PY{p}{,} \PY{l+m+mi}{100}\PY{p}{,} \PY{l+m+mf}{0.2}\PY{p}{)}  

\PY{n}{k} \PY{o}{=} \PY{l+m+mi}{4}
\PY{n}{\PYZus{}}\PY{p}{,} \PY{n}{labels}\PY{p}{,} \PY{p}{(}\PY{n}{centers}\PY{p}{)} \PY{o}{=} \PY{n}{cv2}\PY{o}{.}\PY{n}{kmeans}\PY{p}{(}\PY{n}{pixel\PYZus{}values}\PY{p}{,} \PY{n}{k}\PY{p}{,} \PY{k+kc}{None}\PY{p}{,} \PY{n}{criteria}\PY{p}{,} \PY{l+m+mi}{10}\PY{p}{,} \PY{n}{cv2}\PY{o}{.}\PY{n}{KMEANS\PYZus{}RANDOM\PYZus{}CENTERS}\PY{p}{)} 

\PY{c+c1}{\PYZsh{}Преобразование обратно в 8\PYZhy{}битные значения}
\PY{n}{centers} \PY{o}{=} \PY{n}{np}\PY{o}{.}\PY{n}{uint8}\PY{p}{(}\PY{n}{centers}\PY{p}{)} 

\PY{c+c1}{\PYZsh{}Сгладить массив меток}
\PY{n}{labels} \PY{o}{=} \PY{n}{labels}\PY{o}{.}\PY{n}{flatten}\PY{p}{(}\PY{p}{)} 

\PY{n}{centers} \PY{c+c1}{\PYZsh{}Средний цвет RGB каждого кластера}
\end{Verbatim}
\end{tcolorbox}

            \begin{tcolorbox}[breakable, size=fbox, boxrule=.5pt, pad at break*=1mm, opacityfill=0]
\prompt{Out}{outcolor}{65}{\boxspacing}
\begin{Verbatim}[commandchars=\\\{\}]
array([[ 91,  10,   0],
       [144,  89,  11],
       [ 46,   0,   0],
       [231, 221, 192]], dtype=uint8)
\end{Verbatim}
\end{tcolorbox}
        
    \begin{tcolorbox}[breakable, size=fbox, boxrule=1pt, pad at break*=1mm,colback=cellbackground, colframe=cellborder]
\prompt{In}{incolor}{66}{\boxspacing}
\begin{Verbatim}[commandchars=\\\{\}]
\PY{c+c1}{\PYZsh{}labels }
\PY{c+c1}{\PYZsh{}labels.shape}

\PY{n}{segmented\PYZus{}image} \PY{o}{=} \PY{n}{centers}\PY{p}{[}\PY{n}{labels}\PY{o}{.}\PY{n}{flatten}\PY{p}{(}\PY{p}{)}\PY{p}{]}
\PY{c+c1}{\PYZsh{}segmented\PYZus{}image }

\PY{c+c1}{\PYZsh{}Изменить форму до исходного размера изображения}
\PY{n}{segmented\PYZus{}image} \PY{o}{=} \PY{n}{segmented\PYZus{}image}\PY{o}{.}\PY{n}{reshape}\PY{p}{(}\PY{n}{bilayerImg}\PY{o}{.}\PY{n}{shape}\PY{p}{)}

\PY{c+c1}{\PYZsh{}Выводим изображение}
\PY{n}{fig} \PY{o}{=} \PY{n}{plt}\PY{o}{.}\PY{n}{figure}\PY{p}{(}\PY{p}{)}
\PY{n}{plt}\PY{o}{.}\PY{n}{imshow}\PY{p}{(}\PY{n}{segmented\PYZus{}image}\PY{p}{)}
\PY{n}{plt}\PY{o}{.}\PY{n}{axis}\PY{p}{(}\PY{l+s+s1}{\PYZsq{}}\PY{l+s+s1}{off}\PY{l+s+s1}{\PYZsq{}}\PY{p}{)}
\PY{n}{fig}\PY{o}{.}\PY{n}{savefig}\PY{p}{(}\PY{l+s+s1}{\PYZsq{}}\PY{l+s+s1}{supp\PYZus{}Fig\PYZus{}1aa.png}\PY{l+s+s1}{\PYZsq{}}\PY{p}{,}\PY{n}{dpi} \PY{o}{=}\PY{l+m+mi}{1200}\PY{p}{)}
\end{Verbatim}
\end{tcolorbox}

    \begin{center}
    \adjustimage{max size={0.9\linewidth}{0.9\paperheight}}{output_19_0.png}
    \end{center}
    { \hspace*{\fill} \\}
    
    \begin{tcolorbox}[breakable, size=fbox, boxrule=1pt, pad at break*=1mm,colback=cellbackground, colframe=cellborder]
\prompt{In}{incolor}{77}{\boxspacing}
\begin{Verbatim}[commandchars=\\\{\}]
\PY{c+c1}{\PYZsh{}Частотный график для каждого кластера}

\PY{n}{pd}\PY{o}{.}\PY{n}{Series}\PY{p}{(}\PY{n}{labels}\PY{p}{)}\PY{o}{.}\PY{n}{value\PYZus{}counts}\PY{p}{(}\PY{p}{)}\PY{o}{/}\PY{n}{labels}\PY{o}{.}\PY{n}{shape}\PY{p}{[}\PY{l+m+mi}{0}\PY{p}{]}\PY{o}{*}\PY{l+m+mi}{100}

\PY{n}{propor} \PY{o}{=} \PY{n}{pd}\PY{o}{.}\PY{n}{Series}\PY{p}{(}\PY{n}{labels}\PY{p}{)}\PY{o}{.}\PY{n}{value\PYZus{}counts}\PY{p}{(}\PY{p}{)}\PY{o}{/}\PY{n}{labels}\PY{o}{.}\PY{n}{shape}\PY{p}{[}\PY{l+m+mi}{0}\PY{p}{]}\PY{o}{*}\PY{l+m+mi}{100}

\PY{n}{plt}\PY{o}{.}\PY{n}{rcParams}\PY{o}{.}\PY{n}{update}\PY{p}{(}\PY{p}{\PYZob{}}\PY{l+s+s1}{\PYZsq{}}\PY{l+s+s1}{font.size}\PY{l+s+s1}{\PYZsq{}}\PY{p}{:} \PY{l+m+mi}{16}\PY{p}{,} \PY{l+s+s1}{\PYZsq{}}\PY{l+s+s1}{font.family}\PY{l+s+s1}{\PYZsq{}}\PY{p}{:} \PY{l+s+s1}{\PYZsq{}}\PY{l+s+s1}{STIXGeneral}\PY{l+s+s1}{\PYZsq{}}\PY{p}{,} \PY{l+s+s1}{\PYZsq{}}\PY{l+s+s1}{mathtext.fontset}\PY{l+s+s1}{\PYZsq{}}\PY{p}{:} \PY{l+s+s1}{\PYZsq{}}\PY{l+s+s1}{stix}\PY{l+s+s1}{\PYZsq{}}\PY{p}{\PYZcb{}}\PY{p}{)}

\PY{n}{fig} \PY{o}{=} \PY{n}{plt}\PY{o}{.}\PY{n}{figure}\PY{p}{(}\PY{p}{)}
\PY{n}{bar} \PY{o}{=} \PY{n}{plt}\PY{o}{.}\PY{n}{bar}\PY{p}{(}\PY{p}{[}\PY{l+s+s1}{\PYZsq{}}\PY{l+s+s1}{1}\PY{l+s+s1}{\PYZsq{}}\PY{p}{,}\PY{l+s+s1}{\PYZsq{}}\PY{l+s+s1}{2}\PY{l+s+s1}{\PYZsq{}}\PY{p}{,}\PY{l+s+s1}{\PYZsq{}}\PY{l+s+s1}{3}\PY{l+s+s1}{\PYZsq{}}\PY{p}{,}\PY{l+s+s1}{\PYZsq{}}\PY{l+s+s1}{4}\PY{l+s+s1}{\PYZsq{}}\PY{p}{]}\PY{p}{,}\PY{n}{propor}\PY{o}{.}\PY{n}{values}\PY{p}{)}
\PY{n}{bar}\PY{p}{[}\PY{l+m+mi}{0}\PY{p}{]}\PY{o}{.}\PY{n}{set\PYZus{}color}\PY{p}{(}\PY{p}{(}\PY{l+m+mi}{91}\PY{o}{/}\PY{l+m+mi}{255}\PY{p}{,}\PY{l+m+mi}{10}\PY{o}{/}\PY{l+m+mi}{255}\PY{p}{,}\PY{l+m+mi}{0}\PY{p}{)}\PY{p}{)}
\PY{n}{bar}\PY{p}{[}\PY{l+m+mi}{1}\PY{p}{]}\PY{o}{.}\PY{n}{set\PYZus{}color}\PY{p}{(}\PY{p}{(}\PY{l+m+mi}{46}\PY{o}{/}\PY{l+m+mi}{255}\PY{p}{,}\PY{l+m+mi}{0}\PY{p}{,}\PY{l+m+mi}{0}\PY{p}{)}\PY{p}{)}
\PY{n}{bar}\PY{p}{[}\PY{l+m+mi}{2}\PY{p}{]}\PY{o}{.}\PY{n}{set\PYZus{}color}\PY{p}{(}\PY{p}{(}\PY{l+m+mi}{144}\PY{o}{/}\PY{l+m+mi}{255}\PY{p}{,}\PY{l+m+mi}{88}\PY{o}{/}\PY{l+m+mi}{255}\PY{p}{,}\PY{l+m+mi}{11}\PY{o}{/}\PY{l+m+mi}{255}\PY{p}{)}\PY{p}{)}
\PY{n}{bar}\PY{p}{[}\PY{l+m+mi}{3}\PY{p}{]}\PY{o}{.}\PY{n}{set\PYZus{}color}\PY{p}{(}\PY{p}{(}\PY{l+m+mi}{231}\PY{o}{/}\PY{l+m+mi}{255}\PY{p}{,}\PY{l+m+mi}{221}\PY{o}{/}\PY{l+m+mi}{255}\PY{p}{,}\PY{l+m+mi}{191}\PY{o}{/}\PY{l+m+mi}{255}\PY{p}{)}\PY{p}{)}
\PY{n}{plt}\PY{o}{.}\PY{n}{ylabel}\PY{p}{(}\PY{l+s+s1}{\PYZsq{}}\PY{l+s+s1}{Частота}\PY{l+s+s1}{\PYZsq{}}\PY{p}{)}
\PY{n}{plt}\PY{o}{.}\PY{n}{xlabel}\PY{p}{(}\PY{l+s+s1}{\PYZsq{}}\PY{l+s+s1}{Домены}\PY{l+s+s1}{\PYZsq{}}\PY{p}{)}
\PY{n}{fig}\PY{o}{.}\PY{n}{savefig}\PY{p}{(}\PY{l+s+s1}{\PYZsq{}}\PY{l+s+s1}{supp\PYZus{}Fig\PYZus{}1b.png}\PY{l+s+s1}{\PYZsq{}}\PY{p}{,}\PY{n}{dpi} \PY{o}{=}\PY{l+m+mi}{1200}\PY{p}{)}
\end{Verbatim}
\end{tcolorbox}

    \begin{center}
    \adjustimage{max size={0.9\linewidth}{0.9\paperheight}}{output_20_0.png}
    \end{center}
    { \hspace*{\fill} \\}
    
    \begin{tcolorbox}[breakable, size=fbox, boxrule=1pt, pad at break*=1mm,colback=cellbackground, colframe=cellborder]
\prompt{In}{incolor}{68}{\boxspacing}
\begin{Verbatim}[commandchars=\\\{\}]
\PY{n}{centers} \PY{c+c1}{\PYZsh{}Средний цвет RGB каждого кластера}
\end{Verbatim}
\end{tcolorbox}

            \begin{tcolorbox}[breakable, size=fbox, boxrule=.5pt, pad at break*=1mm, opacityfill=0]
\prompt{Out}{outcolor}{68}{\boxspacing}
\begin{Verbatim}[commandchars=\\\{\}]
array([[ 91,  10,   0],
       [144,  89,  11],
       [ 46,   0,   0],
       [231, 221, 192]], dtype=uint8)
\end{Verbatim}
\end{tcolorbox}
        
    \begin{tcolorbox}[breakable, size=fbox, boxrule=1pt, pad at break*=1mm,colback=cellbackground, colframe=cellborder]
\prompt{In}{incolor}{69}{\boxspacing}
\begin{Verbatim}[commandchars=\\\{\}]
\PY{c+c1}{\PYZsh{}Самый яркий домен}

\PY{n}{c} \PY{o}{=} \PY{l+m+mi}{0}
\PY{n}{meann} \PY{o}{=} \PY{p}{[}\PY{p}{]}
\PY{k}{for} \PY{n}{i} \PY{o+ow}{in} \PY{n}{scaleImg}\PY{p}{:}
    \PY{n}{c} \PY{o}{+}\PY{o}{=} \PY{l+m+mi}{1}
    \PY{k}{if} \PY{n}{np}\PY{o}{.}\PY{n}{sum}\PY{p}{(}\PY{n+nb}{abs}\PY{p}{(}\PY{n}{i} \PY{o}{\PYZhy{}} \PY{n}{np}\PY{o}{.}\PY{n}{array}\PY{p}{(}\PY{n}{centers}\PY{p}{[}\PY{l+m+mi}{0}\PY{p}{]}\PY{p}{)}\PY{p}{)}\PY{p}{)} \PY{o}{\PYZlt{}} \PY{l+m+mi}{9}\PY{p}{:}
      \PY{n+nb}{print}\PY{p}{(}\PY{l+s+s1}{\PYZsq{}}\PY{l+s+s1}{Match}\PY{l+s+s1}{\PYZsq{}}\PY{p}{,}\PY{n}{c}\PY{p}{)}
      \PY{n}{meann}\PY{o}{.}\PY{n}{append}\PY{p}{(}\PY{n}{c}\PY{p}{)}
\end{Verbatim}
\end{tcolorbox}

    \begin{Verbatim}[commandchars=\\\{\}]
Match 589
Match 590
Match 591
Match 592
    \end{Verbatim}

    \begin{tcolorbox}[breakable, size=fbox, boxrule=1pt, pad at break*=1mm,colback=cellbackground, colframe=cellborder]
\prompt{In}{incolor}{70}{\boxspacing}
\begin{Verbatim}[commandchars=\\\{\}]
\PY{c+c1}{\PYZsh{}Средняя высота самого яркого домена}

\PY{n}{Bilayer\PYZus{}h} \PY{o}{=} \PY{l+m+mf}{14.6} \PY{o}{+} \PY{l+m+mf}{9.7}  \PY{c+c1}{\PYZsh{}Высота микрофотографии АСМ}
\PY{n}{Match} \PY{o}{=} \PY{l+m+mi}{102}             \PY{c+c1}{\PYZsh{}Индекс средней высоты}
\PY{n}{Scale\PYZus{}pixls} \PY{o}{=} \PY{l+m+mi}{1020}      \PY{c+c1}{\PYZsh{}Количество пикселей в красной линии }
\PY{n}{Top\PYZus{}h} \PY{o}{=} \PY{l+m+mf}{14.6}            \PY{c+c1}{\PYZsh{}Самая высокая точка на микрофотографии ACM}

\PY{n}{h} \PY{o}{=} \PY{p}{(}\PY{n}{Match}\PY{o}{*}\PY{n}{Bilayer\PYZus{}h}\PY{p}{)}\PY{o}{/}\PY{n}{Scale\PYZus{}pixls}
\PY{n}{Real\PYZus{}h} \PY{o}{=} \PY{n}{Top\PYZus{}h}\PY{o}{\PYZhy{}}\PY{n}{h}
\PY{n+nb}{print}\PY{p}{(}\PY{n}{Real\PYZus{}h}\PY{p}{)}
\end{Verbatim}
\end{tcolorbox}

    \begin{Verbatim}[commandchars=\\\{\}]
12.17
    \end{Verbatim}

    \begin{tcolorbox}[breakable, size=fbox, boxrule=1pt, pad at break*=1mm,colback=cellbackground, colframe=cellborder]
\prompt{In}{incolor}{71}{\boxspacing}
\begin{Verbatim}[commandchars=\\\{\}]
\PY{c+c1}{\PYZsh{}The bright domain}

\PY{n}{c} \PY{o}{=} \PY{l+m+mi}{0}
\PY{n}{meann} \PY{o}{=} \PY{p}{[}\PY{p}{]}
\PY{k}{for} \PY{n}{i} \PY{o+ow}{in} \PY{n}{scaleImg}\PY{p}{:}
    \PY{n}{c} \PY{o}{+}\PY{o}{=} \PY{l+m+mi}{1}
    \PY{k}{if} \PY{n}{np}\PY{o}{.}\PY{n}{sum}\PY{p}{(}\PY{n+nb}{abs}\PY{p}{(}\PY{n}{i} \PY{o}{\PYZhy{}} \PY{n}{np}\PY{o}{.}\PY{n}{array}\PY{p}{(}\PY{n}{centers}\PY{p}{[}\PY{l+m+mi}{1}\PY{p}{]}\PY{p}{)}\PY{p}{)}\PY{p}{)} \PY{o}{\PYZlt{}} \PY{l+m+mi}{35}\PY{p}{:}
      \PY{n+nb}{print}\PY{p}{(}\PY{l+s+s1}{\PYZsq{}}\PY{l+s+s1}{Match}\PY{l+s+s1}{\PYZsq{}}\PY{p}{,}\PY{n}{c}\PY{p}{)}
      \PY{n}{meann}\PY{o}{.}\PY{n}{append}\PY{p}{(}\PY{n}{c}\PY{p}{)}
\end{Verbatim}
\end{tcolorbox}

    \begin{Verbatim}[commandchars=\\\{\}]
Match 370
Match 371
Match 372
Match 373
Match 374
Match 379
Match 380
Match 401
    \end{Verbatim}

    \begin{tcolorbox}[breakable, size=fbox, boxrule=1pt, pad at break*=1mm,colback=cellbackground, colframe=cellborder]
\prompt{In}{incolor}{72}{\boxspacing}
\begin{Verbatim}[commandchars=\\\{\}]
\PY{c+c1}{\PYZsh{}Средняя высота яркого домена}

\PY{n}{Bilayer\PYZus{}h} \PY{o}{=} \PY{l+m+mf}{14.6} \PY{o}{+} \PY{l+m+mf}{9.7}  \PY{c+c1}{\PYZsh{}Высота микрофотографии АСМ}
\PY{n}{Match} \PY{o}{=} \PY{l+m+mi}{371}             \PY{c+c1}{\PYZsh{}Индекс средней высоты}
\PY{n}{Scale\PYZus{}pixls} \PY{o}{=} \PY{l+m+mi}{1020}      \PY{c+c1}{\PYZsh{}Количество пикселей в красной линии }
\PY{n}{Top\PYZus{}h} \PY{o}{=} \PY{l+m+mf}{14.6}            \PY{c+c1}{\PYZsh{}Самая высокая точка на микрофотографии ACM}

\PY{n}{h} \PY{o}{=} \PY{p}{(}\PY{n}{Match}\PY{o}{*}\PY{n}{Bilayer\PYZus{}h}\PY{p}{)}\PY{o}{/}\PY{n}{Scale\PYZus{}pixls}
\PY{n}{Real\PYZus{}h} \PY{o}{=} \PY{n}{Top\PYZus{}h}\PY{o}{\PYZhy{}}\PY{n}{h}
\PY{n+nb}{print}\PY{p}{(}\PY{n}{Real\PYZus{}h}\PY{p}{)}
\end{Verbatim}
\end{tcolorbox}

    \begin{Verbatim}[commandchars=\\\{\}]
5.761470588235294
    \end{Verbatim}

    \begin{tcolorbox}[breakable, size=fbox, boxrule=1pt, pad at break*=1mm,colback=cellbackground, colframe=cellborder]
\prompt{In}{incolor}{73}{\boxspacing}
\begin{Verbatim}[commandchars=\\\{\}]
\PY{c+c1}{\PYZsh{}The dark domain}

\PY{n}{c} \PY{o}{=} \PY{l+m+mi}{0}
\PY{n}{meann} \PY{o}{=} \PY{p}{[}\PY{p}{]}
\PY{k}{for} \PY{n}{i} \PY{o+ow}{in} \PY{n}{scaleImg}\PY{p}{:}
    \PY{n}{c} \PY{o}{+}\PY{o}{=} \PY{l+m+mi}{1}
    \PY{k}{if} \PY{n}{np}\PY{o}{.}\PY{n}{sum}\PY{p}{(}\PY{n+nb}{abs}\PY{p}{(}\PY{n}{i} \PY{o}{\PYZhy{}} \PY{n}{np}\PY{o}{.}\PY{n}{array}\PY{p}{(}\PY{n}{centers}\PY{p}{[}\PY{l+m+mi}{2}\PY{p}{]}\PY{p}{)}\PY{p}{)}\PY{p}{)} \PY{o}{\PYZlt{}} \PY{l+m+mi}{7}\PY{p}{:}
      \PY{n+nb}{print}\PY{p}{(}\PY{l+s+s1}{\PYZsq{}}\PY{l+s+s1}{Match}\PY{l+s+s1}{\PYZsq{}}\PY{p}{,}\PY{n}{c}\PY{p}{)}
      \PY{n}{meann}\PY{o}{.}\PY{n}{append}\PY{p}{(}\PY{n}{c}\PY{p}{)}
\end{Verbatim}
\end{tcolorbox}

    \begin{Verbatim}[commandchars=\\\{\}]
Match 765
Match 766
Match 767
Match 768
Match 769
Match 770
Match 771
Match 772
    \end{Verbatim}

    \begin{tcolorbox}[breakable, size=fbox, boxrule=1pt, pad at break*=1mm,colback=cellbackground, colframe=cellborder]
\prompt{In}{incolor}{74}{\boxspacing}
\begin{Verbatim}[commandchars=\\\{\}]
\PY{c+c1}{\PYZsh{}Mean height of the dark domain}

\PY{n}{Bilayer\PYZus{}h} \PY{o}{=} \PY{l+m+mf}{14.6} \PY{o}{+} \PY{l+m+mf}{9.7}  \PY{c+c1}{\PYZsh{}Height of the AFM micrograph}
\PY{n}{Match} \PY{o}{=} \PY{l+m+mi}{770}             \PY{c+c1}{\PYZsh{}Index of the mean height}
\PY{n}{Scale\PYZus{}pixls} \PY{o}{=} \PY{l+m+mi}{1020}      \PY{c+c1}{\PYZsh{}Amount of pixels in the red line }
\PY{n}{Top\PYZus{}h} \PY{o}{=} \PY{l+m+mf}{14.6}            \PY{c+c1}{\PYZsh{}The highest point in the AFM micrograph}

\PY{n}{h} \PY{o}{=} \PY{p}{(}\PY{n}{Match}\PY{o}{*}\PY{n}{Bilayer\PYZus{}h}\PY{p}{)}\PY{o}{/}\PY{n}{Scale\PYZus{}pixls}
\PY{n}{Real\PYZus{}h} \PY{o}{=} \PY{n}{Top\PYZus{}h}\PY{o}{\PYZhy{}}\PY{n}{h}
\PY{n+nb}{print}\PY{p}{(}\PY{n}{Real\PYZus{}h}\PY{p}{)}
\end{Verbatim}
\end{tcolorbox}

    \begin{Verbatim}[commandchars=\\\{\}]
-3.74411764705882
    \end{Verbatim}

    \begin{tcolorbox}[breakable, size=fbox, boxrule=1pt, pad at break*=1mm,colback=cellbackground, colframe=cellborder]
\prompt{In}{incolor}{75}{\boxspacing}
\begin{Verbatim}[commandchars=\\\{\}]
\PY{c+c1}{\PYZsh{}Фоновый домен}

\PY{n}{c} \PY{o}{=} \PY{l+m+mi}{0}
\PY{n}{meann} \PY{o}{=} \PY{p}{[}\PY{p}{]}
\PY{k}{for} \PY{n}{i} \PY{o+ow}{in} \PY{n}{scaleImg}\PY{p}{:}
    \PY{n}{c} \PY{o}{+}\PY{o}{=} \PY{l+m+mi}{1}
    \PY{k}{if} \PY{n}{np}\PY{o}{.}\PY{n}{sum}\PY{p}{(}\PY{n+nb}{abs}\PY{p}{(}\PY{n}{i} \PY{o}{\PYZhy{}} \PY{n}{np}\PY{o}{.}\PY{n}{array}\PY{p}{(}\PY{n}{centers}\PY{p}{[}\PY{l+m+mi}{3}\PY{p}{]}\PY{p}{)}\PY{p}{)}\PY{p}{)} \PY{o}{\PYZlt{}} \PY{l+m+mi}{7}\PY{p}{:}
      \PY{n+nb}{print}\PY{p}{(}\PY{l+s+s1}{\PYZsq{}}\PY{l+s+s1}{Match}\PY{l+s+s1}{\PYZsq{}}\PY{p}{,}\PY{n}{c}\PY{p}{)}
      \PY{n}{meann}\PY{o}{.}\PY{n}{append}\PY{p}{(}\PY{n}{c}\PY{p}{)}
\end{Verbatim}
\end{tcolorbox}

    \begin{Verbatim}[commandchars=\\\{\}]
Match 104
Match 105
Match 106
Match 107
Match 108
    \end{Verbatim}

    \begin{tcolorbox}[breakable, size=fbox, boxrule=1pt, pad at break*=1mm,colback=cellbackground, colframe=cellborder]
\prompt{In}{incolor}{76}{\boxspacing}
\begin{Verbatim}[commandchars=\\\{\}]
\PY{c+c1}{\PYZsh{}Средняя высота фонового домена}

\PY{n}{Bilayer\PYZus{}h} \PY{o}{=} \PY{l+m+mf}{14.6} \PY{o}{+} \PY{l+m+mf}{9.7}  \PY{c+c1}{\PYZsh{}Высота микрофотографии АСМ}
\PY{n}{Match} \PY{o}{=} \PY{l+m+mi}{591}             \PY{c+c1}{\PYZsh{}Индекс средней высоты}
\PY{n}{Scale\PYZus{}pixls} \PY{o}{=} \PY{l+m+mi}{1020}      \PY{c+c1}{\PYZsh{}Количество пикселей в красной линии }
\PY{n}{Top\PYZus{}h} \PY{o}{=} \PY{l+m+mf}{14.6}            \PY{c+c1}{\PYZsh{}Самая высокая точка на микрофотографии ACM}

\PY{n}{h} \PY{o}{=} \PY{p}{(}\PY{n}{Match}\PY{o}{*}\PY{n}{Bilayer\PYZus{}h}\PY{p}{)}\PY{o}{/}\PY{n}{Scale\PYZus{}pixls}
\PY{n}{Real\PYZus{}h} \PY{o}{=} \PY{n}{Top\PYZus{}h}\PY{o}{\PYZhy{}}\PY{n}{h}
\PY{n+nb}{print}\PY{p}{(}\PY{n}{Real\PYZus{}h}\PY{p}{)}
\end{Verbatim}
\end{tcolorbox}

    \begin{Verbatim}[commandchars=\\\{\}]
0.5202941176470617
    \end{Verbatim}

    \begin{tcolorbox}[breakable, size=fbox, boxrule=1pt, pad at break*=1mm,colback=cellbackground, colframe=cellborder]
\prompt{In}{incolor}{ }{\boxspacing}
\begin{Verbatim}[commandchars=\\\{\}]

\end{Verbatim}
\end{tcolorbox}

    \begin{tcolorbox}[breakable, size=fbox, boxrule=1pt, pad at break*=1mm,colback=cellbackground, colframe=cellborder]
\prompt{In}{incolor}{ }{\boxspacing}
\begin{Verbatim}[commandchars=\\\{\}]

\end{Verbatim}
\end{tcolorbox}

    \begin{tcolorbox}[breakable, size=fbox, boxrule=1pt, pad at break*=1mm,colback=cellbackground, colframe=cellborder]
\prompt{In}{incolor}{ }{\boxspacing}
\begin{Verbatim}[commandchars=\\\{\}]

\end{Verbatim}
\end{tcolorbox}


\section{Заключение}
В результате практической работы был продемонстрирован на практике метод измерения толщины LB пленки с помощью библиотеки компьютерного зрения OpenCV.

Пленки LB были предложены для многих практических применений, охватывающих широкий спектр различных областей от биодатчиков до просветляющих покрытий и полностью оптических процессоров. Были достигнуты значительные успехи в конструировании молекул и LB-структур с функциями, и взаимосвязь между макроскопическими и молекулярными структурами была рассмотрена очень подробно. Тем временем было достигнуто четкое понимание основных принципов, регулирующих упаковку и структуру пленки LB. Однако, несмотря на этот значительный прогресс, пленки LB еще не нашли своего выхода на рынок, а конкурентные подходы оказались более успешными с коммерческой точки зрения. Тем не менее, методы LB остаются потенциально полезными из-за своей простоты и способности создавать молекулярные сборки, определенные на молекулярном уровне. В этом отношении пленки LB будут продолжать служить ценными модельными системами для решения основных научных проблем, таких как смачиваемость, трение и молекулярное распознавание, поэтому, полученный практический опыт является актуальным и может быть применен в дальнейшей научной и производственной деятельности.

\begin{thebibliography}{3}    
\bibitem{Chanci}	Katerine Chanci, Johnatan Diosa, Marco A. Giraldo, Monica Mesa,
Physical behavior of KR-12 peptide on solid surfaces and Langmuir-Blodgett lipid films: Complementary approaches to its antimicrobial mode against S. aureus, Biochimica et Biophysica Acta (BBA) - Biomembranes, Volume 1864, Issue 1,2022,183779,ISSN 0005-2736, ~\url{https://doi.org/10.1016/j.bbamem.2021.183779}.
    
\bibitem{Handbook} Handbook of Applied Surface and Colloid Chemistry. Edited by Krister Holmberg, 2001 John Wiley and Sons, Ltd
    
\bibitem{c1}	Roberts, G., Langmuir-Blodgett Films, Plenum Press, New York, 1990.
\bibitem{c2}	Roberts, G. G., An applied science perspective of Langmuir-Blodgett Films, Adv. Phys., 34, 475-512 (1985).
\bibitem{c3}	Peterson, I. R., Langmuir-Blodgett techniques, in The Molecular Electronic Handbook , Mahler, G., May, V. and Schreiber, M. (Eds), Marcel Dekker, New York, 1994, pp. 47-77.
\bibitem{c4}	Gaines, G. L., Insoluble Monolayers at Liquid-Gas Inter- faces, Wiley-Interscience, New York, 1966.
\bibitem{c5}	Ulmann, A., Introduction to Ultrathin Organic Films: From Langmuir-Blodgett to Self-Assembly, Academic Press, San Diego, CA, 1991.
\bibitem{c6}	Kaganer, M., Peterson, I. R., Kenn, R., Shih, M., Durbin, M. and Dutta, P., Tiltod phoses of fatty acid mono- layers, J. Chem. Phys., 102, 9412-9422 (1995).
\bibitem{c7}	Mohwald, H., Phospholipid and phospholipid-protein monolayers at the air/water interface, Annu. Rev. Phys. Chem., 41, 441-••• (1990).
\bibitem{c8}	McConnel, H. M., Structures and transitions in lipid monolayers at the air - water interface, Annu. Rev. Phys. Chem., 42, 171-205 (1991).
\bibitem{c9}	Knobler, C. M. and Desai, R. C., Phase transition in monolayers, Annu. Rev. Phys. Chem., 43, 207 - 236 (1991).
\bibitem{c10}	Riviere, S., Henon, S., Meunier, J., Schwartz, D. K., Tsao, M. W. and Knobler, C. M., Texture and phase transitions in Langmuir monolayers of fatty acids - a comparative Brewster angle microscope and polarized fluorescence study, J. Chem. Phys., 101, 10045 - 10051 (1994).
\bibitem{c11}	Knobler, C. and Schwartz D., Langmuir and self-assembled monolayers, Curr. Opinion Colloid Interface Sci., 4, 46-51 (1999).

\end{thebibliography}    
    
\end{document}
