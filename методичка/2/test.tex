\documentclass[11pt]{article}
	
    \usepackage[breakable]{tcolorbox}
    \usepackage{parskip} % Stop auto-indenting (to mimic markdown behaviour)
    
    \usepackage{iftex}
    \ifPDFTeX
    	\usepackage[T1]{fontenc}
    	\usepackage{mathpazo}
    \else
    	\usepackage{fontspec}
    \fi

    % Basic figure setup, for now with no caption control since it's done
    % automatically by Pandoc (which extracts ![](path) syntax from Markdown).
    \usepackage{graphicx}
    % Maintain compatibility with old templates. Remove in nbconvert 6.0
    \let\Oldincludegraphics\includegraphics
    % Ensure that by default, figures have no caption (until we provide a
    % proper Figure object with a Caption API and a way to capture that
    % in the conversion process - todo).
    \usepackage{caption}
    \DeclareCaptionFormat{nocaption}{}
    \captionsetup{format=nocaption,aboveskip=0pt,belowskip=0pt}

    \usepackage{float}
    \floatplacement{figure}{H} % forces figures to be placed at the correct location
    \usepackage{xcolor} % Allow colors to be defined
    \usepackage{enumerate} % Needed for markdown enumerations to work
    \usepackage{geometry} % Used to adjust the document margins
    \usepackage{amsmath} % Equations
    \usepackage{amssymb} % Equations
    \usepackage{textcomp} % defines textquotesingle
    % Hack from http://tex.stackexchange.com/a/47451/13684:
    \AtBeginDocument{%
        \def\PYZsq{\textquotesingle}% Upright quotes in Pygmentized code
    }
    \usepackage{upquote} % Upright quotes for verbatim code
    \usepackage{eurosym} % defines \euro
    \usepackage[mathletters]{ucs} % Extended unicode (utf-8) support
    \usepackage{fancyvrb} % verbatim replacement that allows latex
    \usepackage{grffile} % extends the file name processing of package graphics 
                         % to support a larger range
    \makeatletter % fix for old versions of grffile with XeLaTeX
    \@ifpackagelater{grffile}{2019/11/01}
    {
      % Do nothing on new versions
    }
    {
      \def\Gread@@xetex#1{%
        \IfFileExists{"\Gin@base".bb}%
        {\Gread@eps{\Gin@base.bb}}%
        {\Gread@@xetex@aux#1}%
      }
    }
    \makeatother
    \usepackage[Export]{adjustbox} % Used to constrain images to a maximum size
    \adjustboxset{max size={0.9\linewidth}{0.9\paperheight}}

    % The hyperref package gives us a pdf with properly built
    % internal navigation ('pdf bookmarks' for the table of contents,
    % internal cross-reference links, web links for URLs, etc.)
    \usepackage{hyperref}
    % The default LaTeX title has an obnoxious amount of whitespace. By default,
    % titling removes some of it. It also provides customization options.
    \usepackage{titling}
    \usepackage{longtable} % longtable support required by pandoc >1.10
    \usepackage{booktabs}  % table support for pandoc > 1.12.2
    \usepackage[inline]{enumitem} % IRkernel/repr support (it uses the enumerate* environment)
    \usepackage[normalem]{ulem} % ulem is needed to support strikethroughs (\sout)
                                % normalem makes italics be italics, not underlines
    \usepackage{mathrsfs}
    
  	\usepackage{ucs} 
	\usepackage[utf8x]{inputenc} % Включаем поддержку UTF8  
	\usepackage[russian]{babel}  % Включаем пакет для поддержки русского языка 
    

    
    % Colors for the hyperref package
    \definecolor{urlcolor}{rgb}{0,.145,.698}
    \definecolor{linkcolor}{rgb}{.71,0.21,0.01}
    \definecolor{citecolor}{rgb}{.12,.54,.11}

    % ANSI colors
    \definecolor{ansi-black}{HTML}{3E424D}
    \definecolor{ansi-black-intense}{HTML}{282C36}
    \definecolor{ansi-red}{HTML}{E75C58}
    \definecolor{ansi-red-intense}{HTML}{B22B31}
    \definecolor{ansi-green}{HTML}{00A250}
    \definecolor{ansi-green-intense}{HTML}{007427}
    \definecolor{ansi-yellow}{HTML}{DDB62B}
    \definecolor{ansi-yellow-intense}{HTML}{B27D12}
    \definecolor{ansi-blue}{HTML}{208FFB}
    \definecolor{ansi-blue-intense}{HTML}{0065CA}
    \definecolor{ansi-magenta}{HTML}{D160C4}
    \definecolor{ansi-magenta-intense}{HTML}{A03196}
    \definecolor{ansi-cyan}{HTML}{60C6C8}
    \definecolor{ansi-cyan-intense}{HTML}{258F8F}
    \definecolor{ansi-white}{HTML}{C5C1B4}
    \definecolor{ansi-white-intense}{HTML}{A1A6B2}
    \definecolor{ansi-default-inverse-fg}{HTML}{FFFFFF}
    \definecolor{ansi-default-inverse-bg}{HTML}{000000}

    % common color for the border for error outputs.
    \definecolor{outerrorbackground}{HTML}{FFDFDF}

    % commands and environments needed by pandoc snippets
    % extracted from the output of `pandoc -s`
    \providecommand{\tightlist}{%
      \setlength{\itemsep}{0pt}\setlength{\parskip}{0pt}}
    \DefineVerbatimEnvironment{Highlighting}{Verbatim}{commandchars=\\\{\}}
    % Add ',fontsize=\small' for more characters per line
    \newenvironment{Shaded}{}{}
    \newcommand{\KeywordTok}[1]{\textcolor[rgb]{0.00,0.44,0.13}{\textbf{{#1}}}}
    \newcommand{\DataTypeTok}[1]{\textcolor[rgb]{0.56,0.13,0.00}{{#1}}}
    \newcommand{\DecValTok}[1]{\textcolor[rgb]{0.25,0.63,0.44}{{#1}}}
    \newcommand{\BaseNTok}[1]{\textcolor[rgb]{0.25,0.63,0.44}{{#1}}}
    \newcommand{\FloatTok}[1]{\textcolor[rgb]{0.25,0.63,0.44}{{#1}}}
    \newcommand{\CharTok}[1]{\textcolor[rgb]{0.25,0.44,0.63}{{#1}}}
    \newcommand{\StringTok}[1]{\textcolor[rgb]{0.25,0.44,0.63}{{#1}}}
    \newcommand{\CommentTok}[1]{\textcolor[rgb]{0.38,0.63,0.69}{\textit{{#1}}}}
    \newcommand{\OtherTok}[1]{\textcolor[rgb]{0.00,0.44,0.13}{{#1}}}
    \newcommand{\AlertTok}[1]{\textcolor[rgb]{1.00,0.00,0.00}{\textbf{{#1}}}}
    \newcommand{\FunctionTok}[1]{\textcolor[rgb]{0.02,0.16,0.49}{{#1}}}
    \newcommand{\RegionMarkerTok}[1]{{#1}}
    \newcommand{\ErrorTok}[1]{\textcolor[rgb]{1.00,0.00,0.00}{\textbf{{#1}}}}
    \newcommand{\NormalTok}[1]{{#1}}
    
    % Additional commands for more recent versions of Pandoc
    \newcommand{\ConstantTok}[1]{\textcolor[rgb]{0.53,0.00,0.00}{{#1}}}
    \newcommand{\SpecialCharTok}[1]{\textcolor[rgb]{0.25,0.44,0.63}{{#1}}}
    \newcommand{\VerbatimStringTok}[1]{\textcolor[rgb]{0.25,0.44,0.63}{{#1}}}
    \newcommand{\SpecialStringTok}[1]{\textcolor[rgb]{0.73,0.40,0.53}{{#1}}}
    \newcommand{\ImportTok}[1]{{#1}}
    \newcommand{\DocumentationTok}[1]{\textcolor[rgb]{0.73,0.13,0.13}{\textit{{#1}}}}
    \newcommand{\AnnotationTok}[1]{\textcolor[rgb]{0.38,0.63,0.69}{\textbf{\textit{{#1}}}}}
    \newcommand{\CommentVarTok}[1]{\textcolor[rgb]{0.38,0.63,0.69}{\textbf{\textit{{#1}}}}}
    \newcommand{\VariableTok}[1]{\textcolor[rgb]{0.10,0.09,0.49}{{#1}}}
    \newcommand{\ControlFlowTok}[1]{\textcolor[rgb]{0.00,0.44,0.13}{\textbf{{#1}}}}
    \newcommand{\OperatorTok}[1]{\textcolor[rgb]{0.40,0.40,0.40}{{#1}}}
    \newcommand{\BuiltInTok}[1]{{#1}}
    \newcommand{\ExtensionTok}[1]{{#1}}
    \newcommand{\PreprocessorTok}[1]{\textcolor[rgb]{0.74,0.48,0.00}{{#1}}}
    \newcommand{\AttributeTok}[1]{\textcolor[rgb]{0.49,0.56,0.16}{{#1}}}
    \newcommand{\InformationTok}[1]{\textcolor[rgb]{0.38,0.63,0.69}{\textbf{\textit{{#1}}}}}
    \newcommand{\WarningTok}[1]{\textcolor[rgb]{0.38,0.63,0.69}{\textbf{\textit{{#1}}}}}
    
    
    % Define a nice break command that doesn't care if a line doesn't already
    % exist.
    \def\br{\hspace*{\fill} \\* }
    % Math Jax compatibility definitions
    \def\gt{>}
    \def\lt{<}
    \let\Oldtex\TeX
    \let\Oldlatex\LaTeX
    \renewcommand{\TeX}{\textrm{\Oldtex}}
    \renewcommand{\LaTeX}{\textrm{\Oldlatex}}
    % Document parameters
    % Document title
    \title{Практическое занятие 2.1 Сочленения в опорно-двигательном аппарате человека}
    
    
    
    
    
% Pygments definitions
\makeatletter
\def\PY@reset{\let\PY@it=\relax \let\PY@bf=\relax%
    \let\PY@ul=\relax \let\PY@tc=\relax%
    \let\PY@bc=\relax \let\PY@ff=\relax}
\def\PY@tok#1{\csname PY@tok@#1\endcsname}
\def\PY@toks#1+{\ifx\relax#1\empty\else%
    \PY@tok{#1}\expandafter\PY@toks\fi}
\def\PY@do#1{\PY@bc{\PY@tc{\PY@ul{%
    \PY@it{\PY@bf{\PY@ff{#1}}}}}}}
\def\PY#1#2{\PY@reset\PY@toks#1+\relax+\PY@do{#2}}

\@namedef{PY@tok@w}{\def\PY@tc##1{\textcolor[rgb]{0.73,0.73,0.73}{##1}}}
\@namedef{PY@tok@c}{\let\PY@it=\textit\def\PY@tc##1{\textcolor[rgb]{0.24,0.48,0.48}{##1}}}
\@namedef{PY@tok@cp}{\def\PY@tc##1{\textcolor[rgb]{0.61,0.40,0.00}{##1}}}
\@namedef{PY@tok@k}{\let\PY@bf=\textbf\def\PY@tc##1{\textcolor[rgb]{0.00,0.50,0.00}{##1}}}
\@namedef{PY@tok@kp}{\def\PY@tc##1{\textcolor[rgb]{0.00,0.50,0.00}{##1}}}
\@namedef{PY@tok@kt}{\def\PY@tc##1{\textcolor[rgb]{0.69,0.00,0.25}{##1}}}
\@namedef{PY@tok@o}{\def\PY@tc##1{\textcolor[rgb]{0.40,0.40,0.40}{##1}}}
\@namedef{PY@tok@ow}{\let\PY@bf=\textbf\def\PY@tc##1{\textcolor[rgb]{0.67,0.13,1.00}{##1}}}
\@namedef{PY@tok@nb}{\def\PY@tc##1{\textcolor[rgb]{0.00,0.50,0.00}{##1}}}
\@namedef{PY@tok@nf}{\def\PY@tc##1{\textcolor[rgb]{0.00,0.00,1.00}{##1}}}
\@namedef{PY@tok@nc}{\let\PY@bf=\textbf\def\PY@tc##1{\textcolor[rgb]{0.00,0.00,1.00}{##1}}}
\@namedef{PY@tok@nn}{\let\PY@bf=\textbf\def\PY@tc##1{\textcolor[rgb]{0.00,0.00,1.00}{##1}}}
\@namedef{PY@tok@ne}{\let\PY@bf=\textbf\def\PY@tc##1{\textcolor[rgb]{0.80,0.25,0.22}{##1}}}
\@namedef{PY@tok@nv}{\def\PY@tc##1{\textcolor[rgb]{0.10,0.09,0.49}{##1}}}
\@namedef{PY@tok@no}{\def\PY@tc##1{\textcolor[rgb]{0.53,0.00,0.00}{##1}}}
\@namedef{PY@tok@nl}{\def\PY@tc##1{\textcolor[rgb]{0.46,0.46,0.00}{##1}}}
\@namedef{PY@tok@ni}{\let\PY@bf=\textbf\def\PY@tc##1{\textcolor[rgb]{0.44,0.44,0.44}{##1}}}
\@namedef{PY@tok@na}{\def\PY@tc##1{\textcolor[rgb]{0.41,0.47,0.13}{##1}}}
\@namedef{PY@tok@nt}{\let\PY@bf=\textbf\def\PY@tc##1{\textcolor[rgb]{0.00,0.50,0.00}{##1}}}
\@namedef{PY@tok@nd}{\def\PY@tc##1{\textcolor[rgb]{0.67,0.13,1.00}{##1}}}
\@namedef{PY@tok@s}{\def\PY@tc##1{\textcolor[rgb]{0.73,0.13,0.13}{##1}}}
\@namedef{PY@tok@sd}{\let\PY@it=\textit\def\PY@tc##1{\textcolor[rgb]{0.73,0.13,0.13}{##1}}}
\@namedef{PY@tok@si}{\let\PY@bf=\textbf\def\PY@tc##1{\textcolor[rgb]{0.64,0.35,0.47}{##1}}}
\@namedef{PY@tok@se}{\let\PY@bf=\textbf\def\PY@tc##1{\textcolor[rgb]{0.67,0.36,0.12}{##1}}}
\@namedef{PY@tok@sr}{\def\PY@tc##1{\textcolor[rgb]{0.64,0.35,0.47}{##1}}}
\@namedef{PY@tok@ss}{\def\PY@tc##1{\textcolor[rgb]{0.10,0.09,0.49}{##1}}}
\@namedef{PY@tok@sx}{\def\PY@tc##1{\textcolor[rgb]{0.00,0.50,0.00}{##1}}}
\@namedef{PY@tok@m}{\def\PY@tc##1{\textcolor[rgb]{0.40,0.40,0.40}{##1}}}
\@namedef{PY@tok@gh}{\let\PY@bf=\textbf\def\PY@tc##1{\textcolor[rgb]{0.00,0.00,0.50}{##1}}}
\@namedef{PY@tok@gu}{\let\PY@bf=\textbf\def\PY@tc##1{\textcolor[rgb]{0.50,0.00,0.50}{##1}}}
\@namedef{PY@tok@gd}{\def\PY@tc##1{\textcolor[rgb]{0.63,0.00,0.00}{##1}}}
\@namedef{PY@tok@gi}{\def\PY@tc##1{\textcolor[rgb]{0.00,0.52,0.00}{##1}}}
\@namedef{PY@tok@gr}{\def\PY@tc##1{\textcolor[rgb]{0.89,0.00,0.00}{##1}}}
\@namedef{PY@tok@ge}{\let\PY@it=\textit}
\@namedef{PY@tok@gs}{\let\PY@bf=\textbf}
\@namedef{PY@tok@gp}{\let\PY@bf=\textbf\def\PY@tc##1{\textcolor[rgb]{0.00,0.00,0.50}{##1}}}
\@namedef{PY@tok@go}{\def\PY@tc##1{\textcolor[rgb]{0.44,0.44,0.44}{##1}}}
\@namedef{PY@tok@gt}{\def\PY@tc##1{\textcolor[rgb]{0.00,0.27,0.87}{##1}}}
\@namedef{PY@tok@err}{\def\PY@bc##1{{\setlength{\fboxsep}{\string -\fboxrule}\fcolorbox[rgb]{1.00,0.00,0.00}{1,1,1}{\strut ##1}}}}
\@namedef{PY@tok@kc}{\let\PY@bf=\textbf\def\PY@tc##1{\textcolor[rgb]{0.00,0.50,0.00}{##1}}}
\@namedef{PY@tok@kd}{\let\PY@bf=\textbf\def\PY@tc##1{\textcolor[rgb]{0.00,0.50,0.00}{##1}}}
\@namedef{PY@tok@kn}{\let\PY@bf=\textbf\def\PY@tc##1{\textcolor[rgb]{0.00,0.50,0.00}{##1}}}
\@namedef{PY@tok@kr}{\let\PY@bf=\textbf\def\PY@tc##1{\textcolor[rgb]{0.00,0.50,0.00}{##1}}}
\@namedef{PY@tok@bp}{\def\PY@tc##1{\textcolor[rgb]{0.00,0.50,0.00}{##1}}}
\@namedef{PY@tok@fm}{\def\PY@tc##1{\textcolor[rgb]{0.00,0.00,1.00}{##1}}}
\@namedef{PY@tok@vc}{\def\PY@tc##1{\textcolor[rgb]{0.10,0.09,0.49}{##1}}}
\@namedef{PY@tok@vg}{\def\PY@tc##1{\textcolor[rgb]{0.10,0.09,0.49}{##1}}}
\@namedef{PY@tok@vi}{\def\PY@tc##1{\textcolor[rgb]{0.10,0.09,0.49}{##1}}}
\@namedef{PY@tok@vm}{\def\PY@tc##1{\textcolor[rgb]{0.10,0.09,0.49}{##1}}}
\@namedef{PY@tok@sa}{\def\PY@tc##1{\textcolor[rgb]{0.73,0.13,0.13}{##1}}}
\@namedef{PY@tok@sb}{\def\PY@tc##1{\textcolor[rgb]{0.73,0.13,0.13}{##1}}}
\@namedef{PY@tok@sc}{\def\PY@tc##1{\textcolor[rgb]{0.73,0.13,0.13}{##1}}}
\@namedef{PY@tok@dl}{\def\PY@tc##1{\textcolor[rgb]{0.73,0.13,0.13}{##1}}}
\@namedef{PY@tok@s2}{\def\PY@tc##1{\textcolor[rgb]{0.73,0.13,0.13}{##1}}}
\@namedef{PY@tok@sh}{\def\PY@tc##1{\textcolor[rgb]{0.73,0.13,0.13}{##1}}}
\@namedef{PY@tok@s1}{\def\PY@tc##1{\textcolor[rgb]{0.73,0.13,0.13}{##1}}}
\@namedef{PY@tok@mb}{\def\PY@tc##1{\textcolor[rgb]{0.40,0.40,0.40}{##1}}}
\@namedef{PY@tok@mf}{\def\PY@tc##1{\textcolor[rgb]{0.40,0.40,0.40}{##1}}}
\@namedef{PY@tok@mh}{\def\PY@tc##1{\textcolor[rgb]{0.40,0.40,0.40}{##1}}}
\@namedef{PY@tok@mi}{\def\PY@tc##1{\textcolor[rgb]{0.40,0.40,0.40}{##1}}}
\@namedef{PY@tok@il}{\def\PY@tc##1{\textcolor[rgb]{0.40,0.40,0.40}{##1}}}
\@namedef{PY@tok@mo}{\def\PY@tc##1{\textcolor[rgb]{0.40,0.40,0.40}{##1}}}
\@namedef{PY@tok@ch}{\let\PY@it=\textit\def\PY@tc##1{\textcolor[rgb]{0.24,0.48,0.48}{##1}}}
\@namedef{PY@tok@cm}{\let\PY@it=\textit\def\PY@tc##1{\textcolor[rgb]{0.24,0.48,0.48}{##1}}}
\@namedef{PY@tok@cpf}{\let\PY@it=\textit\def\PY@tc##1{\textcolor[rgb]{0.24,0.48,0.48}{##1}}}
\@namedef{PY@tok@c1}{\let\PY@it=\textit\def\PY@tc##1{\textcolor[rgb]{0.24,0.48,0.48}{##1}}}
\@namedef{PY@tok@cs}{\let\PY@it=\textit\def\PY@tc##1{\textcolor[rgb]{0.24,0.48,0.48}{##1}}}

\def\PYZbs{\char`\\}
\def\PYZus{\char`\_}
\def\PYZob{\char`\{}
\def\PYZcb{\char`\}}
\def\PYZca{\char`\^}
\def\PYZam{\char`\&}
\def\PYZlt{\char`\<}
\def\PYZgt{\char`\>}
\def\PYZsh{\char`\#}
\def\PYZpc{\char`\%}
\def\PYZdl{\char`\$}
\def\PYZhy{\char`\-}
\def\PYZsq{\char`\'}
\def\PYZdq{\char`\"}
\def\PYZti{\char`\~}
% for compatibility with earlier versions
\def\PYZat{@}
\def\PYZlb{[}
\def\PYZrb{]}
\makeatother


    % For linebreaks inside Verbatim environment from package fancyvrb. 
    \makeatletter
        \newbox\Wrappedcontinuationbox 
        \newbox\Wrappedvisiblespacebox 
        \newcommand*\Wrappedvisiblespace {\textcolor{red}{\textvisiblespace}} 
        \newcommand*\Wrappedcontinuationsymbol {\textcolor{red}{\llap{\tiny$\m@th\hookrightarrow$}}} 
        \newcommand*\Wrappedcontinuationindent {3ex } 
        \newcommand*\Wrappedafterbreak {\kern\Wrappedcontinuationindent\copy\Wrappedcontinuationbox} 
        % Take advantage of the already applied Pygments mark-up to insert 
        % potential linebreaks for TeX processing. 
        %        {, <, #, %, $, ' and ": go to next line. 
        %        _, }, ^, &, >, - and ~: stay at end of broken line. 
        % Use of \textquotesingle for straight quote. 
        \newcommand*\Wrappedbreaksatspecials {% 
            \def\PYGZus{\discretionary{\char`\_}{\Wrappedafterbreak}{\char`\_}}% 
            \def\PYGZob{\discretionary{}{\Wrappedafterbreak\char`\{}{\char`\{}}% 
            \def\PYGZcb{\discretionary{\char`\}}{\Wrappedafterbreak}{\char`\}}}% 
            \def\PYGZca{\discretionary{\char`\^}{\Wrappedafterbreak}{\char`\^}}% 
            \def\PYGZam{\discretionary{\char`\&}{\Wrappedafterbreak}{\char`\&}}% 
            \def\PYGZlt{\discretionary{}{\Wrappedafterbreak\char`\<}{\char`\<}}% 
            \def\PYGZgt{\discretionary{\char`\>}{\Wrappedafterbreak}{\char`\>}}% 
            \def\PYGZsh{\discretionary{}{\Wrappedafterbreak\char`\#}{\char`\#}}% 
            \def\PYGZpc{\discretionary{}{\Wrappedafterbreak\char`\%}{\char`\%}}% 
            \def\PYGZdl{\discretionary{}{\Wrappedafterbreak\char`\$}{\char`\$}}% 
            \def\PYGZhy{\discretionary{\char`\-}{\Wrappedafterbreak}{\char`\-}}% 
            \def\PYGZsq{\discretionary{}{\Wrappedafterbreak\textquotesingle}{\textquotesingle}}% 
            \def\PYGZdq{\discretionary{}{\Wrappedafterbreak\char`\"}{\char`\"}}% 
            \def\PYGZti{\discretionary{\char`\~}{\Wrappedafterbreak}{\char`\~}}% 
        } 
        % Some characters . , ; ? ! / are not pygmentized. 
        % This macro makes them "active" and they will insert potential linebreaks 
        \newcommand*\Wrappedbreaksatpunct {% 
            \lccode`\~`\.\lowercase{\def~}{\discretionary{\hbox{\char`\.}}{\Wrappedafterbreak}{\hbox{\char`\.}}}% 
            \lccode`\~`\,\lowercase{\def~}{\discretionary{\hbox{\char`\,}}{\Wrappedafterbreak}{\hbox{\char`\,}}}% 
            \lccode`\~`\;\lowercase{\def~}{\discretionary{\hbox{\char`\;}}{\Wrappedafterbreak}{\hbox{\char`\;}}}% 
            \lccode`\~`\:\lowercase{\def~}{\discretionary{\hbox{\char`\:}}{\Wrappedafterbreak}{\hbox{\char`\:}}}% 
            \lccode`\~`\?\lowercase{\def~}{\discretionary{\hbox{\char`\?}}{\Wrappedafterbreak}{\hbox{\char`\?}}}% 
            \lccode`\~`\!\lowercase{\def~}{\discretionary{\hbox{\char`\!}}{\Wrappedafterbreak}{\hbox{\char`\!}}}% 
            \lccode`\~`\/\lowercase{\def~}{\discretionary{\hbox{\char`\/}}{\Wrappedafterbreak}{\hbox{\char`\/}}}% 
            \catcode`\.\active
            \catcode`\,\active 
            \catcode`\;\active
            \catcode`\:\active
            \catcode`\?\active
            \catcode`\!\active
            \catcode`\/\active 
            \lccode`\~`\~ 	
        }
    \makeatother

    \let\OriginalVerbatim=\Verbatim
    \makeatletter
    \renewcommand{\Verbatim}[1][1]{%
        %\parskip\z@skip
        \sbox\Wrappedcontinuationbox {\Wrappedcontinuationsymbol}%
        \sbox\Wrappedvisiblespacebox {\FV@SetupFont\Wrappedvisiblespace}%
        \def\FancyVerbFormatLine ##1{\hsize\linewidth
            \vtop{\raggedright\hyphenpenalty\z@\exhyphenpenalty\z@
                \doublehyphendemerits\z@\finalhyphendemerits\z@
                \strut ##1\strut}%
        }%
        % If the linebreak is at a space, the latter will be displayed as visible
        % space at end of first line, and a continuation symbol starts next line.
        % Stretch/shrink are however usually zero for typewriter font.
        \def\FV@Space {%
            \nobreak\hskip\z@ plus\fontdimen3\font minus\fontdimen4\font
            \discretionary{\copy\Wrappedvisiblespacebox}{\Wrappedafterbreak}
            {\kern\fontdimen2\font}%
        }%
        
        % Allow breaks at special characters using \PYG... macros.
        \Wrappedbreaksatspecials
        % Breaks at punctuation characters . , ; ? ! and / need catcode=\active 	
        \OriginalVerbatim[#1,codes*=\Wrappedbreaksatpunct]%
    }
    \makeatother

    % Exact colors from NB
    \definecolor{incolor}{HTML}{303F9F}
    \definecolor{outcolor}{HTML}{D84315}
    \definecolor{cellborder}{HTML}{CFCFCF}
    \definecolor{cellbackground}{HTML}{F7F7F7}
    
    % prompt
    \makeatletter
    \newcommand{\boxspacing}{\kern\kvtcb@left@rule\kern\kvtcb@boxsep}
    \makeatother
    \newcommand{\prompt}[4]{
        {\ttfamily\llap{{\color{#2}[#3]:\hspace{3pt}#4}}\vspace{-\baselineskip}}
    }
    

    
    % Prevent overflowing lines due to hard-to-break entities
    \sloppy 
    % Setup hyperref package
    \hypersetup{
      breaklinks=true,  % so long urls are correctly broken across lines
      colorlinks=true,
      urlcolor=urlcolor,
      linkcolor=linkcolor,
      citecolor=citecolor,
      }
    % Slightly bigger margins than the latex defaults
    
    \geometry{verbose,tmargin=1in,bmargin=1in,lmargin=1in,rmargin=1in}
    
    

\begin{document}
    
    \maketitle
    
\tableofcontents{}
    
\section{Введение}
Лабораторная работа по теме 2.1 "Сочленения и рычаги в опорно-двигательном аппарате человека" курса М1.Б.4 "Современные проблемы биомедицинской и экологической инженерии" для подготовки магистров по направлению 12.04.04 "Биотехнические системы и технологии".

На лекциях была рассмотрена тема анализа влияния дегенерации диска на механическое поведение поясничного двигательного сегмента с использованием метода конечных элементов. Была проиллюстрирована элементная сетка двигательного сегмента L3/L4 со здоровым и сильно дегенерированным дисками.
Практическое занятие позволяет на практике освоить моделирование и исследование диска методом конечных элементов. 


\section{Цель}
Получить навыки применения практических данных в построении графического материала для визуализации научных данных и результатов расчетов для углубления компетенции представлять современную научную картину мира, выявлять естественнонаучную сущность проблемы, формулировать задачи, определять пути их решения и оценивать эффективность выбора и методов правовой защиты результатов интеллектуальной деятельности с учетом исследований, разработки и проектирования биотехнических систем и технологий.

\section{Модель}
Решаемый пример соответствует определению усилий в цилиндре в бразильском тесте . Бразильский тест — это метод, который используется для косвенного измерения сопротивления горных пород. Это простой и эффективный метод, и поэтому он обычно используется для измерения горных пород. Иногда это испытание также используется для бетона ~\cite{D3967-16} и он так же полезен в области биоматериалов для моделирования и исследования поведения материалов под нагрузкой в сочленениях: областей костей, связок и суставов.  

На следующем рисунке представлена схема модели.
Благодаря симметрии мы решаем задачу для сегмента. Для диска, который под нагрузкой может деформироваться в эллипс оправданно взять сегмент в 90 градусов. Соответственно, для моделирования сферы можно взять 90 градусов телесного угла, при этом алгоритм остается прежним.

    \begin{center}
    \adjustimage{max size={0.9\linewidth}{0.9\paperheight}}{model.png}
    \end{center}

\section{Практическая часть - программирование в Jupyter Notebook}
    \subsection{Подготовка окружения выполнения}
Для успешного выполнения кода на своем компьютере необходимо выполнить ряд подготовительных мероприятий.

\subsubsection{Установка Anaconda}
Anaconda - дистрибутив языков программирования Python и R, включающий набор популярных свободных библиотек, объединённых проблематиками науки о данных и машинного обучения.

Для скачивания пакета Anaconda надо перейти по ссылке и скачать дистрибутив для своей операционной системы:

~\url{https://www.anaconda.com/products/individual}

    \begin{center}
    \adjustimage{max size={0.9\linewidth}{0.9\paperheight}}{1.png}
    \end{center}

После установки необходимо запустить навигатор. В Windows и MacOS X его можно найти в списке приложений, в Linux его можно запустить  из консоли 
    \begin{Verbatim}[commandchars=\\\{\}]
./anaconda-navigator
    \end{Verbatim}
     
Окно навигатора показано на рисунке ниже.
    \begin{center}
    \adjustimage{max size={0.9\linewidth}{0.9\paperheight}}{2.png}
    \end{center}

\subsubsection{Установка зависимостей}
После установки Anaconda нам надо установить дополнительные библиотеки, для этого в консоли необходимо выполнить следующие команды. Их выполнение может занять значительное время, поэтому не стоит их прерывать из-за предположения, что они зависли. Если у вас возникнут сложности, поскольку со временем что то может измениться, то всегда можно обратиться с поиском в интернет и "загуглить" ошибку и способ ее устранения.

    \begin{Verbatim}[commandchars=\\\{\}]
conda update --all --yes
conda install -c conda-forge google-colab
conda install -c conda-forge opencv
conda install -c conda-forge pandoc
conda install -c conda-forge nbconvert
conda install -c conda-forge pyppeteer
pip install solidspy
pip install easygui
pip install meshio
    \end{Verbatim}

\subsubsection{Установка Gmsh}
Gmsh — это генератор 3D-сетки конечных элементов с открытым исходным кодом со встроенным движком САПР и постпроцессором. Цель его разработки — предоставить быстрый, легкий и удобный инструмент создания сетки с параметрическим вводом и расширенными возможностями визуализации. Gmsh построен вокруг четырех модулей: геометрия, сетка, решатель и постобработка. 

Для скачивания пакета Anaconda надо перейти по ссылке и скачать дистрибутив для своей операционной системы:

~\url{https://gmsh.info/}

\subsection{Пошаговая инструкция}
    \subsubsection{Создание геометрии в Gmsh}
    В качестве первого шага предлагается создать новый файл в Gmsh, как показано на следующем рисунке.

    \begin{center}
    \adjustimage{max size={0.9\linewidth}{0.9\paperheight}}{gmsh_1.png}
    \end{center}
    
    Если у вас на каком то шаге работы с Gmsh возникнут проблемы, то можете перейти к следующему разделу и создать файл в текстовом режиме. Gmsh это графический интерфейс, который позволяет сформировать текстовое описание модели, поэтому ничего страшного, если что-то у вас не получится, но если все пройдет в этом разделе "гладко", то следующий раздел "Создание геометрии и сетки в текстовом режиме" можно пропустить.
    
    При создании нового документа Gmsh может [1] задать вопрос о том, какое геометрическое ядро ​​использовать. Не будем останавливаться на том, в чем отличия и воспользуемся built-in.
    
    \begin{center}
    \adjustimage{max size={0.9\linewidth}{0.9\paperheight}}{gmsh_2.png}
    \end{center}    
    
    Для создания модели мы сначала создаем точки. Для этого перейдем к опции, как показано на следующем рисунке. Затем вводим координаты точек во всплывающем окне, нажимаем кнопку «Add» и, наконец, мы можем закрыть всплывающее окно.

    \begin{center}
    \adjustimage{max size={0.9\linewidth}{0.9\paperheight}}{gmsh_3.png}
    \end{center}    
    
    На следующем шаге мы создаем линии.  Для этого переходим к опции: 

\begin{Verbatim}[commandchars=\\\{\}]
Geometry> Elementary Entities> Add> Straight line
\end{Verbatim}
    
    , как показано на следующем рисунке, и выбираем начальные точки и окончания для каждой линии. В конце нужно нажать "e" в английской раскладке.
    
    \begin{center}
    \adjustimage{max size={0.9\linewidth}{0.9\paperheight}}{gmsh_4.png}
    \end{center}    
    
Мы также создаем дуги окружности. Для этого переходим к опции:

\begin{Verbatim}[commandchars=\\\{\}]
Geometry> Elementary Entities> Add> Circle Arc
\end{Verbatim}
 
,как показано на следующем рисунке, и выбираем начальные точки, центральную и конечную для каждой дуги (в таком порядке). В конце нам нужно нажать "e" в английской раскладке.

    \begin{center}
    \adjustimage{max size={0.9\linewidth}{0.9\paperheight}}{gmsh_5.png}
    \end{center} 

Теперь, поскольку у нас уже есть замкнутый контур, мы можем определить поверхность. Для этого переходим к опции: 

\begin{Verbatim}[commandchars=\\\{\}]
Geometry > Elementary Entities > Add > Plane Surface
\end{Verbatim}

, как показано на следующем рисунке, и выбираем контуры по порядку. В конце нам нужно нажать "e" в английской раскладке.
    
    \begin{center}
    \adjustimage{max size={0.9\linewidth}{0.9\paperheight}}{gmsh_6.png}
    \end{center}
        
Теперь нам нужно определить физические группы. Физические группы позволяют нам связывать имена с различными частями модели, такими как линии и поверхности. Это позволит нам определить область, в которой мы будем разрешать модель (и свяжем материал), области с ограниченными движениями (граничные условия) и области, к которым мы применим нагрузку. В нашем случае у нас будет 4 физических группы:
        
\begin{itemize}
  \item Область модели, где мы определим материал;
  \item Нижний край, где мы ограничим смещение по y;
  \item Левый край, где мы ограничим смещение по x;
  \item Верхняя точка, к которой мы будем прикладывать точечную нагрузку.
\end{itemize}     
        
Чтобы определить физические группы, мы идем в 

\begin{Verbatim}[commandchars=\\\{\}]
Geometry> Physical groups> Add> Plane Surface
\end{Verbatim}

, как показано на следующем рисунке. В этом случае мы можем оставить поле «Имя» пустым и позволить Gmsh назвать группы для нас, которые будут числами, которые мы затем можем просмотреть в текстовом файле.
        
\begin{center}
    \adjustimage{max size={0.9\linewidth}{0.9\paperheight}}{gmsh_7.png}
\end{center}
    
 \subsubsection{Создание геометрии в текстовом режиме}
   
   Графическое представление файла (.geo) выглядит так:
   \begin{center}
    \adjustimage{max size={0.9\linewidth}{0.9\paperheight}}{gmsh_8.png}
\end{center}
   
    Текстовый файл (.geo) выглядит так:
    \begin{Verbatim}[commandchars=\\\{\}]
L = 0.1;

// Points
Point(1) = {0, 0, 0, L};
Point(2) = {1, 0, 0, L};
Point(3) = {0, 1, 0, L};

// Lines
Line(1) = {3, 1};
Line(2) = {1, 2};
Circle(3) = {2, 1, 3};

// Surfaces
Line Loop(1) = {2, 3, 1};
Plane Surface(1) = {1};

// Physical groups
Physical Line(1) = {1};
Physical Line(2) = {2};
Physical Point(3) = {3};
Physical Surface(4) = {1};
\end{Verbatim}

Вы можете проверить, что сохраненный файл соответствует листингу и в случае расхождения, поправить файл в любом текстовом редакторе.

 \subsubsection{Создание сетки в Gmsh}
 
Мы добавили параметр L, который мы можем изменить, чтобы изменить размер элементов при создании сетки.

Теперь приступаем к созданию сетки. Для этого переходим в 
\begin{Verbatim}[commandchars=\\\{\}]
Mesh > 2D
\end{Verbatim}

,как мы видим на рисунке ниже:

\begin{center}
    \adjustimage{max size={0.9\linewidth}{0.9\paperheight}}{gmsh_9.png}
\end{center}

\subsubsection{Скрипт Python для создания входных файлов}
Нам необходимо создать файлы с информацией об узлах ( nodes.txt), элементах ( eles.txt), нагрузках ( loads.txt) и материалах ( mater.txt).

Следующий код создает необходимые входные файлы для запуска программы конечных элементов в Python.

    \begin{tcolorbox}[breakable, size=fbox, boxrule=1pt, pad at break*=1mm,colback=cellbackground, colframe=cellborder]
\prompt{In}{incolor}{1}{\boxspacing}
\begin{Verbatim}[commandchars=\\\{\}]
\PY{k+kn}{import} \PY{n+nn}{matplotlib}\PY{n+nn}{.}\PY{n+nn}{pyplot} \PY{k}{as} \PY{n+nn}{plt}  \PY{c+c1}{\PYZsh{} load matplotlib}
\PY{k+kn}{from} \PY{n+nn}{solidspy} \PY{k+kn}{import} \PY{n}{solids\PYZus{}GUI}  \PY{c+c1}{\PYZsh{} import our package}
\PY{n}{disp} \PY{o}{=} \PY{n}{solids\PYZus{}GUI}\PY{p}{(}\PY{p}{)}  \PY{c+c1}{\PYZsh{} run the Finite Element Analysis}
\PY{n}{plt}\PY{o}{.}\PY{n}{show}\PY{p}{(}\PY{p}{)}    \PY{c+c1}{\PYZsh{} plot contours}
\end{Verbatim}
\end{tcolorbox}

    \begin{Verbatim}[commandchars=\\\{\}]
Number of nodes: 9
Number of elements: 4
Number of equations: 14
Duration for system solution: 0:00:00.080328
    \end{Verbatim}

    \begin{Verbatim}[commandchars=\\\{\}]
/home/il1/anaconda3/envs/intro\_fem/lib/python3.9/site-
packages/solidspy/postprocesor.py:108: UserWarning: The following kwargs were
not used by contour: 'shading'
  disp\_plot(tri, field, levels, shading="gouraud")
    \end{Verbatim}

    \begin{Verbatim}[commandchars=\\\{\}]
Duration for post processing: 0:00:00.191287
Analysis terminated successfully!
    \end{Verbatim}

    \begin{center}
    \adjustimage{max size={0.9\linewidth}{0.9\paperheight}}{output_0_3.png}
    \end{center}
    { \hspace*{\fill} \\}
    
    \begin{center}
    \adjustimage{max size={0.9\linewidth}{0.9\paperheight}}{output_0_4.png}
    \end{center}
    { \hspace*{\fill} \\}
    
    \begin{tcolorbox}[breakable, size=fbox, boxrule=1pt, pad at break*=1mm,colback=cellbackground, colframe=cellborder]
\prompt{In}{incolor}{2}{\boxspacing}
\begin{Verbatim}[commandchars=\\\{\}]
\PY{n}{UC} \PY{o}{=} \PY{n}{solids\PYZus{}GUI}\PY{p}{(}\PY{p}{)}
\end{Verbatim}
\end{tcolorbox}

    \begin{Verbatim}[commandchars=\\\{\}]
Number of nodes: 9
Number of elements: 4
Number of equations: 14
Duration for system solution: 0:00:00.030502
Duration for post processing: 0:00:00.136590
Analysis terminated successfully!
    \end{Verbatim}

    \begin{center}
    \adjustimage{max size={0.9\linewidth}{0.9\paperheight}}{output_1_1.png}
    \end{center}
    { \hspace*{\fill} \\}
    
    \begin{center}
    \adjustimage{max size={0.9\linewidth}{0.9\paperheight}}{output_1_2.png}
    \end{center}
    { \hspace*{\fill} \\}
    
    \begin{tcolorbox}[breakable, size=fbox, boxrule=1pt, pad at break*=1mm,colback=cellbackground, colframe=cellborder]
\prompt{In}{incolor}{4}{\boxspacing}
\begin{Verbatim}[commandchars=\\\{\}]
\PY{k+kn}{import} \PY{n+nn}{meshio}
\PY{k+kn}{import} \PY{n+nn}{numpy} \PY{k}{as} \PY{n+nn}{np}

\PY{n}{mesh} \PY{o}{=} \PY{n}{meshio}\PY{o}{.}\PY{n}{read}\PY{p}{(}\PY{l+s+s2}{\PYZdq{}}\PY{l+s+s2}{g.msh}\PY{l+s+s2}{\PYZdq{}}\PY{p}{)}
\PY{n}{points} \PY{o}{=} \PY{n}{mesh}\PY{o}{.}\PY{n}{points}
\PY{n}{cells} \PY{o}{=} \PY{n}{mesh}\PY{o}{.}\PY{n}{cells}
\PY{n}{point\PYZus{}data} \PY{o}{=} \PY{n}{mesh}\PY{o}{.}\PY{n}{point\PYZus{}data}
\PY{n}{cell\PYZus{}data} \PY{o}{=} \PY{n}{mesh}\PY{o}{.}\PY{n}{cell\PYZus{}data}

\PY{c+c1}{\PYZsh{} Element data}
\PY{n}{eles} \PY{o}{=} \PY{n}{cells}\PY{p}{[}\PY{l+s+s2}{\PYZdq{}}\PY{l+s+s2}{triangle}\PY{l+s+s2}{\PYZdq{}}\PY{p}{]}
\PY{n}{els\PYZus{}array} \PY{o}{=} \PY{n}{np}\PY{o}{.}\PY{n}{zeros}\PY{p}{(}\PY{p}{[}\PY{n}{eles}\PY{o}{.}\PY{n}{shape}\PY{p}{[}\PY{l+m+mi}{0}\PY{p}{]}\PY{p}{,} \PY{l+m+mi}{6}\PY{p}{]}\PY{p}{,} \PY{n}{dtype}\PY{o}{=}\PY{n+nb}{int}\PY{p}{)}
\PY{n}{els\PYZus{}array}\PY{p}{[}\PY{p}{:}\PY{p}{,} \PY{l+m+mi}{0}\PY{p}{]} \PY{o}{=} \PY{n+nb}{range}\PY{p}{(}\PY{n}{eles}\PY{o}{.}\PY{n}{shape}\PY{p}{[}\PY{l+m+mi}{0}\PY{p}{]}\PY{p}{)}
\PY{n}{els\PYZus{}array}\PY{p}{[}\PY{p}{:}\PY{p}{,} \PY{l+m+mi}{1}\PY{p}{]} \PY{o}{=} \PY{l+m+mi}{3}
\PY{n}{els\PYZus{}array}\PY{p}{[}\PY{p}{:}\PY{p}{,} \PY{l+m+mi}{3}\PY{p}{:}\PY{p}{:}\PY{p}{]} \PY{o}{=} \PY{n}{eles}

\PY{c+c1}{\PYZsh{} Nodes}
\PY{n}{nodes\PYZus{}array} \PY{o}{=} \PY{n}{np}\PY{o}{.}\PY{n}{zeros}\PY{p}{(}\PY{p}{[}\PY{n}{points}\PY{o}{.}\PY{n}{shape}\PY{p}{[}\PY{l+m+mi}{0}\PY{p}{]}\PY{p}{,} \PY{l+m+mi}{5}\PY{p}{]}\PY{p}{)}
\PY{n}{nodes\PYZus{}array}\PY{p}{[}\PY{p}{:}\PY{p}{,} \PY{l+m+mi}{0}\PY{p}{]} \PY{o}{=} \PY{n+nb}{range}\PY{p}{(}\PY{n}{points}\PY{o}{.}\PY{n}{shape}\PY{p}{[}\PY{l+m+mi}{0}\PY{p}{]}\PY{p}{)}
\PY{n}{nodes\PYZus{}array}\PY{p}{[}\PY{p}{:}\PY{p}{,} \PY{l+m+mi}{1}\PY{p}{:}\PY{l+m+mi}{3}\PY{p}{]} \PY{o}{=} \PY{n}{points}\PY{p}{[}\PY{p}{:}\PY{p}{,} \PY{p}{:}\PY{l+m+mi}{2}\PY{p}{]}

\PY{c+c1}{\PYZsh{} Boundaries}
\PY{n}{lines} \PY{o}{=} \PY{n}{cells}\PY{p}{[}\PY{l+s+s2}{\PYZdq{}}\PY{l+s+s2}{line}\PY{l+s+s2}{\PYZdq{}}\PY{p}{]}
\PY{n}{bounds} \PY{o}{=} \PY{n}{cell\PYZus{}data}\PY{p}{[}\PY{l+s+s2}{\PYZdq{}}\PY{l+s+s2}{line}\PY{l+s+s2}{\PYZdq{}}\PY{p}{]}\PY{p}{[}\PY{l+s+s2}{\PYZdq{}}\PY{l+s+s2}{gmsh:physical}\PY{l+s+s2}{\PYZdq{}}\PY{p}{]}
\PY{n}{nbounds} \PY{o}{=} \PY{n+nb}{len}\PY{p}{(}\PY{n}{bounds}\PY{p}{)}

\PY{c+c1}{\PYZsh{} Loads}
\PY{n}{id\PYZus{}cargas} \PY{o}{=} \PY{n}{cells}\PY{p}{[}\PY{l+s+s2}{\PYZdq{}}\PY{l+s+s2}{vertex}\PY{l+s+s2}{\PYZdq{}}\PY{p}{]}
\PY{n}{nloads} \PY{o}{=} \PY{n+nb}{len}\PY{p}{(}\PY{n}{id\PYZus{}cargas}\PY{p}{)}
\PY{n}{load} \PY{o}{=} \PY{o}{\PYZhy{}}\PY{l+m+mf}{10e8} \PY{c+c1}{\PYZsh{} N/m}
\PY{n}{loads\PYZus{}array} \PY{o}{=} \PY{n}{np}\PY{o}{.}\PY{n}{zeros}\PY{p}{(}\PY{p}{(}\PY{n}{nloads}\PY{p}{,} \PY{l+m+mi}{3}\PY{p}{)}\PY{p}{)}
\PY{n}{loads\PYZus{}array}\PY{p}{[}\PY{p}{:}\PY{p}{,} \PY{l+m+mi}{0}\PY{p}{]} \PY{o}{=} \PY{n}{id\PYZus{}cargas}
\PY{n}{loads\PYZus{}array}\PY{p}{[}\PY{p}{:}\PY{p}{,} \PY{l+m+mi}{1}\PY{p}{]} \PY{o}{=} \PY{l+m+mi}{0}
\PY{n}{loads\PYZus{}array}\PY{p}{[}\PY{p}{:}\PY{p}{,} \PY{l+m+mi}{2}\PY{p}{]} \PY{o}{=} \PY{n}{load}

\PY{c+c1}{\PYZsh{} Boundary conditions}
\PY{n}{id\PYZus{}izq} \PY{o}{=} \PY{p}{[}\PY{n}{cont} \PY{k}{for} \PY{n}{cont} \PY{o+ow}{in} \PY{n+nb}{range}\PY{p}{(}\PY{n}{nbounds}\PY{p}{)} \PY{k}{if} \PY{n}{bounds}\PY{p}{[}\PY{n}{cont}\PY{p}{]} \PY{o}{==} \PY{l+m+mi}{1}\PY{p}{]}
\PY{n}{id\PYZus{}inf} \PY{o}{=} \PY{p}{[}\PY{n}{cont} \PY{k}{for} \PY{n}{cont} \PY{o+ow}{in} \PY{n+nb}{range}\PY{p}{(}\PY{n}{nbounds}\PY{p}{)} \PY{k}{if} \PY{n}{bounds}\PY{p}{[}\PY{n}{cont}\PY{p}{]} \PY{o}{==} \PY{l+m+mi}{2}\PY{p}{]}
\PY{n}{nodes\PYZus{}izq} \PY{o}{=} \PY{n}{lines}\PY{p}{[}\PY{n}{id\PYZus{}izq}\PY{p}{]}
\PY{n}{nodes\PYZus{}izq} \PY{o}{=} \PY{n}{nodes\PYZus{}izq}\PY{o}{.}\PY{n}{flatten}\PY{p}{(}\PY{p}{)}
\PY{n}{nodes\PYZus{}inf} \PY{o}{=} \PY{n}{lines}\PY{p}{[}\PY{n}{id\PYZus{}inf}\PY{p}{]}
\PY{n}{nodes\PYZus{}inf} \PY{o}{=} \PY{n}{nodes\PYZus{}inf}\PY{o}{.}\PY{n}{flatten}\PY{p}{(}\PY{p}{)}
\PY{n}{nodes\PYZus{}array}\PY{p}{[}\PY{n}{nodes\PYZus{}izq}\PY{p}{,} \PY{l+m+mi}{3}\PY{p}{]} \PY{o}{=} \PY{o}{\PYZhy{}}\PY{l+m+mi}{1}
\PY{n}{nodes\PYZus{}array}\PY{p}{[}\PY{n}{nodes\PYZus{}inf}\PY{p}{,} \PY{l+m+mi}{4}\PY{p}{]} \PY{o}{=} \PY{o}{\PYZhy{}}\PY{l+m+mi}{1}

\PY{c+c1}{\PYZsh{}  Materials}
\PY{n}{mater\PYZus{}array} \PY{o}{=} \PY{n}{np}\PY{o}{.}\PY{n}{array}\PY{p}{(}\PY{p}{[}\PY{p}{[}\PY{l+m+mf}{70e9}\PY{p}{,} \PY{l+m+mf}{0.35}\PY{p}{]}\PY{p}{,}
                        \PY{p}{[}\PY{l+m+mf}{70e9}\PY{p}{,} \PY{l+m+mf}{0.35}\PY{p}{]}\PY{p}{]}\PY{p}{)}
\PY{n}{maters} \PY{o}{=} \PY{n}{cell\PYZus{}data}\PY{p}{[}\PY{l+s+s2}{\PYZdq{}}\PY{l+s+s2}{triangle}\PY{l+s+s2}{\PYZdq{}}\PY{p}{]}\PY{p}{[}\PY{l+s+s2}{\PYZdq{}}\PY{l+s+s2}{gmsh:physical}\PY{l+s+s2}{\PYZdq{}}\PY{p}{]}
\PY{n}{els\PYZus{}array}\PY{p}{[}\PY{p}{:}\PY{p}{,} \PY{l+m+mi}{2}\PY{p}{]}  \PY{o}{=} \PY{p}{[}\PY{l+m+mi}{1} \PY{k}{for} \PY{n}{mater} \PY{o+ow}{in} \PY{n}{maters} \PY{k}{if} \PY{n}{mater} \PY{o}{==} \PY{l+m+mi}{4}\PY{p}{]}

\PY{c+c1}{\PYZsh{} Create files}
\PY{n}{np}\PY{o}{.}\PY{n}{savetxt}\PY{p}{(}\PY{l+s+s2}{\PYZdq{}}\PY{l+s+s2}{eles.txt}\PY{l+s+s2}{\PYZdq{}}\PY{p}{,} \PY{n}{els\PYZus{}array}\PY{p}{,} \PY{n}{fmt}\PY{o}{=}\PY{l+s+s2}{\PYZdq{}}\PY{l+s+si}{\PYZpc{}d}\PY{l+s+s2}{\PYZdq{}}\PY{p}{)}
\PY{n}{np}\PY{o}{.}\PY{n}{savetxt}\PY{p}{(}\PY{l+s+s2}{\PYZdq{}}\PY{l+s+s2}{nodes.txt}\PY{l+s+s2}{\PYZdq{}}\PY{p}{,} \PY{n}{nodes\PYZus{}array}\PY{p}{,}
           \PY{n}{fmt}\PY{o}{=}\PY{p}{(}\PY{l+s+s2}{\PYZdq{}}\PY{l+s+si}{\PYZpc{}d}\PY{l+s+s2}{\PYZdq{}}\PY{p}{,} \PY{l+s+s2}{\PYZdq{}}\PY{l+s+si}{\PYZpc{}.4f}\PY{l+s+s2}{\PYZdq{}}\PY{p}{,} \PY{l+s+s2}{\PYZdq{}}\PY{l+s+si}{\PYZpc{}.4f}\PY{l+s+s2}{\PYZdq{}}\PY{p}{,} \PY{l+s+s2}{\PYZdq{}}\PY{l+s+si}{\PYZpc{}d}\PY{l+s+s2}{\PYZdq{}}\PY{p}{,} \PY{l+s+s2}{\PYZdq{}}\PY{l+s+si}{\PYZpc{}d}\PY{l+s+s2}{\PYZdq{}}\PY{p}{)}\PY{p}{)}
\PY{n}{np}\PY{o}{.}\PY{n}{savetxt}\PY{p}{(}\PY{l+s+s2}{\PYZdq{}}\PY{l+s+s2}{loads.txt}\PY{l+s+s2}{\PYZdq{}}\PY{p}{,} \PY{n}{loads\PYZus{}array}\PY{p}{,} \PY{n}{fmt}\PY{o}{=}\PY{p}{(}\PY{l+s+s2}{\PYZdq{}}\PY{l+s+si}{\PYZpc{}d}\PY{l+s+s2}{\PYZdq{}}\PY{p}{,} \PY{l+s+s2}{\PYZdq{}}\PY{l+s+si}{\PYZpc{}.6f}\PY{l+s+s2}{\PYZdq{}}\PY{p}{,} \PY{l+s+s2}{\PYZdq{}}\PY{l+s+si}{\PYZpc{}.6f}\PY{l+s+s2}{\PYZdq{}}\PY{p}{)}\PY{p}{)}
\PY{n}{np}\PY{o}{.}\PY{n}{savetxt}\PY{p}{(}\PY{l+s+s2}{\PYZdq{}}\PY{l+s+s2}{mater.txt}\PY{l+s+s2}{\PYZdq{}}\PY{p}{,} \PY{n}{mater\PYZus{}array}\PY{p}{,} \PY{n}{fmt}\PY{o}{=}\PY{l+s+s2}{\PYZdq{}}\PY{l+s+si}{\PYZpc{}.6f}\PY{l+s+s2}{\PYZdq{}}\PY{p}{)}
\end{Verbatim}
\end{tcolorbox}

\subsubsection{Визуализация модели}

    \begin{tcolorbox}[breakable, size=fbox, boxrule=1pt, pad at break*=1mm,colback=cellbackground, colframe=cellborder]
\prompt{In}{incolor}{5}{\boxspacing}
\begin{Verbatim}[commandchars=\\\{\}]
\PY{k+kn}{from} \PY{n+nn}{solidspy} \PY{k+kn}{import} \PY{n}{solids\PYZus{}GUI}
\PY{n}{disp} \PY{o}{=} \PY{n}{solids\PYZus{}GUI}\PY{p}{(}\PY{p}{)}
\end{Verbatim}
\end{tcolorbox}

    \begin{Verbatim}[commandchars=\\\{\}]
Number of nodes: 119
Number of elements: 200
Number of equations: 216
Duration for system solution: 0:00:00.143867
Duration for post processing: 0:00:00.112697
Analysis terminated successfully!
    \end{Verbatim}

    \begin{center}
    \adjustimage{max size={0.9\linewidth}{0.9\paperheight}}{output_3_1.png}
    \end{center}
    { \hspace*{\fill} \\}
    
    \begin{center}
    \adjustimage{max size={0.9\linewidth}{0.9\paperheight}}{output_3_2.png}
    \end{center}
    { \hspace*{\fill} \\}
    
    \begin{tcolorbox}[breakable, size=fbox, boxrule=1pt, pad at break*=1mm,colback=cellbackground, colframe=cellborder]
\prompt{In}{incolor}{ }{\boxspacing}
\begin{Verbatim}[commandchars=\\\{\}]

\end{Verbatim}
\end{tcolorbox}

\section{Заключение}
В результате практической работы был продемонстрирован на практике метод моделирования с помощью конечных элементов. Более подробную информацию можно почерпнуть из специальной литературы ~\cite{Gmsh2009,Gmsh_tut,Gmsh_scr} и на сайте библиотеки ~\url{https://solidspy.readthedocs.io/en/latest/}.

\begin{thebibliography}{3}
\bibitem{D3967-16} ASTM D3967–16 (2016), Standard Test Method for Splitting Tensile Strength of Intact Rock Core Specimens, ASTM International, ~\url{www.astm.org}.

\bibitem{Gmsh2009}	Geuzaine, Christophe, y Jean-François Remacle (2009), Gmsh: A 3-D finite element mesh generator with built-in pre-and post-processing facilities. International Journal for Numerical Methods in Engineering, 79.11.

\bibitem{Gmsh_tut}	Geuzaine, Christophe, y Jean-François Remacle (2017), Gmsh Official Tutorial. ~\url{http://gmsh.info/doc/texinfo/gmsh.html}.

\bibitem{Gmsh_scr}	Geuzaine, Christophe, y Jean-François Remacle (2017), Gmsh Official Screencasts.  ~\url{http://gmsh.info/screencasts/}.
    
\end{thebibliography}
    
\end{document}
